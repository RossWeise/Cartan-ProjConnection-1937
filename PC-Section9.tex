% !TEX root = [Cartan]-ProjConnection.tex

%  Section completed 26 Aug 2017
\ \\

{\bf 9. The natural frame. --- }
% 
Consider any space with projective connection, defined by the  coordinates $(u^1, u^2, ..., u^n)$ of its various points and by the matrix $(\omega)$. We have seen that there exists an infinity of possible transformations for $(\omega)$, that preserve the properties of the various curves in the space. It is therefore natural to take advantage of this indeterminacy to make the matrix ($\omega$) as simple as possible.

First of all, let us turn our attention to the $\omega^i$. By one of these changes of frame previously defined, these quantities undergo linear transformations with arbitrary coefficients
\begin{eqnarray*}
\overline \omega^i &=& b^i_k \omega^k ;
\end{eqnarray*}
we can thus arrange it in such a way that $\omega^1, \omega^2, ...\omega^n$ take respectively the values $du^1, du^2, ..., du^n$:
\begin{eqnarray*}
 \omega^i &=& du^i\, ;
\end{eqnarray*}
the frames realising this condition will be said to be {\em semi-natural}.

The number of forms on which depend the frames attached to the surface is thus reduced to $n (n + 2) - n = n^2 + n$.

Consider the changes of frame that preserve the above form
of the $\omega^i$. For these changes, we will have
\begin{eqnarray*}
\overline \omega^i &=& \omega^i ,
\end{eqnarray*}
from which it follows that 
\begin{eqnarray*}
b^i_i = 1, \ \ \ \ b^j_i = 0\ \ (i\neq j).
\end{eqnarray*}
The matrix of the $b$ will thus have the form
\begin{eqnarray*}
(\, a^{-1} \,) = \left(
\begin{array}{ccccc}
1 & 0  & 0 & \cdots & 0 \\
b^0_1 &  1 & 0 & \cdots & 0 \\
b^0_2 &  0 & 1 & \cdots & 0 \\
\vdots &  &  &  & \vdots \\
b^0_n &  0  &  0  & \cdots & 1 
\end{array}
\right), 
\end{eqnarray*}
and the relations linking the elements of the two matrices $(a)$ and $(a^{-1})$ show that $(a)$ has the same form:
\begin{eqnarray*}
( a) = \left(
\begin{array}{ccccc}
1 & 0  & 0 & \cdots & 0 \\
a^0_1 &  1 & 0 & \cdots & 0 \\
a^0_2 &  0 & 1 & \cdots & 0 \\
\vdots &  &  &  & \vdots \\
a^0_n &  0  &  0  & \cdots & 1 
\end{array}
\right); 
\end{eqnarray*}
we establish immediately that 
\begin{eqnarray*}
(\, a^{-1} \,) = \left(
\begin{array}{ccccc}
1 & 0  & 0 & \cdots & 0 \\
- a^0_1 &  1 & 0 & \cdots & 0 \\
- a^0_2 &  0 & 1 & \cdots & 0 \\
\vdots &  &  &  & \vdots \\
- a^0_n &  0  &  0  & \cdots & 1 
\end{array}
\right). 
\end{eqnarray*}

Consider now the forms $\overline \omega^i_i$ obtained by taking into account the above specialisations. According to the general expression for the $\overline \omega^j_i$ given in paragraph 8, we have 
\begin{eqnarray*}
\overline \omega^1_1 &=& a^0_1\, du^1 + \omega^1_1 + a^0_k\, du^k
\\
\overline \omega^2_2 &=& a^0_2\, du^2 + \omega^2_2 + a^0_k\, du^k
\\
\vdots\ \   &=& \hspace*{1.6cm} \vdots
\\
\overline \omega^n_n &=& a^0_n\, du^n + \omega^n_n + a^0_k\, du^k.
\end{eqnarray*}
By adding these relations we obtain ($i$ being here a summation index)
\begin{eqnarray*}
\overline \omega^i_i &=&  \omega^i_i + (n+1) a^0_k\,, du^k\, :
\end{eqnarray*}
$\omega^i_i$ reproduces itself augmented by the linear form $(n+1) a^0_k\, du^k$; since this form is arbitrary, by the arbitrary character f the $a^0_k$, we can annihilate them. This operation reduces by one the number of arbitrary forms $\omega^\beta_\alpha$, which reduce finally to 
$$  n^2 + n -1. $$
After this new specialisation of the forms $\omega^\beta_\alpha$, on which depends the infinitesimal displacement of the frame, we will have 
\begin{eqnarray*}
(n+1) a^0_k\,, du^k &=& 0,
\end{eqnarray*}
from which 
\begin{eqnarray*}
a^0_k &=& 0,
\end{eqnarray*}
and the matric of the $a$ is reduced to the identity matrix
\begin{eqnarray*}
(a) &=& I .
\end{eqnarray*}

The most general projective connection that can be established on an $n$-dimensional continuum $(u^1, u^2, ..., u^n)$ depends, as we see, on being given $n^2+n-1$ arbitrary forms. Since each of these forms involves $n$ arbitrary functions of the coordinates $u^i$ (the coefficients of the differentials of the coordinates $u^i$), we can say that the most general projective connection that we can attribute to a
given continuum of $n$ dimensions depends on $n(n^2+n-1)$  arbitrary functions of coordinates.

The elements of the matrix associated with each connection are defined by formulas of the type
\begin{eqnarray}
\left.
\begin{array}{rcl}
 \omega^j_i & =  &  \Pi^j_{ik}\, du^k, \\
  &   &  \hspace*{2cm} (\Pi^{ik}_j = 0 ). \\
 \omega^0_i & =  &  \Pi^0_{ik}\, du^k,
\end{array}
\right\}
\label{eq:1-8}
\end{eqnarray}
Once the connection is chosen (the functions $\Pi$ have been fixed),  the motion of the frame attached to any point of the space is determined; the frame that formulas (8) attach abstractly to each  point of the space with projective connection considered is that which  we shall call the {\em natural frame} for that point.
























































































% section complete







































