% EngCguide.tex
% for the suite of standard Cambridge designs
% 2011/02/03, v1.10

  \NeedsTeXFormat{LaTeX2e}[1996/06/01]

  \documentclass{EngC}
% \documentclass[multi]{EngC} % multi-contributor option
% \documentclass[prodtf]{EngC}% if you have helvetica neue condensed fonts

  \usepackage[rightcaption,raggedright]{sidecap}% for side captions
  \usepackage{framed}         % for floatingboxes
  \usepackage{soul}           % for letterspacing in theorem-style headings

% for the Harvard author-date referencing system
  \usepackage[agsm]{harvard}

% if you are using either vancouver.bst or IEEEtran.bst and wish to remove
% square braces in the reference list, uncomment the line below
% \removesquarebraces

  \usepackage{rotating}
  \usepackage{floatpag}
  \rotfloatpagestyle{empty}

% \usepackage{amsmath}% if you are using this package,
                      % it must be loaded before amsthm.sty
  \usepackage{amsthm}
  \usepackage{amssymb}
  \usepackage{amsmath}
  \usepackage{graphicx}
  \usepackage{tikz}
  \usepackage{bm}
  \usepackage{hyperref}

  \usepackage{ffsymb,bold,ffSTVector}

% \usepackage{txfonts}% times font (used to produce EngCguide.pdf)
                      % this package must be loaded after amsthm.sty

\def \vxi {{\vec{\xi}}}


% indexes
% uncomment the relevant set of commands

% for a single index
% \usepackage{makeidx}
% \makeindex

% for multiple indexes using multind.sty
% \usepackage{multind}\ProvidesPackage{multind}
%  \makeindex{authors}
 % \makeindex{subject}

% for multiple indexes using index.sty
% \usepackage{index}
% \newindex{aut}{adx}{and}{Author index}
% \makeindex

  \newcommand\cambridge{EngC}

% see chapter 3 for details
  \theoremstyle{plain}% default
  \newtheorem{theorem}{Theorem}[chapter]
  \newtheorem{lemma}[theorem]{Lemma}
  \newtheorem*{corollary}{Corollary}

  \theoremstyle{definition}
  \newtheorem{definition}[theorem]{Definition}
  \newtheorem{condition}[theorem]{Condition}

  \theoremstyle{remark}
  \newtheorem{remark}{Remark}[chapter]
  \newtheorem{notation}[remark]{Notation}
  \newtheorem*{case}{Case}

\theoremstyle{remark}
  \newtheorem{prob}[theorem]{Problem}

  \theoremstyle{example}
  \newtheorem{example}[theorem]{Example}
%  \newtheorem{condition}[theorem]{Condition}



  \hyphenation{line-break line-breaks docu-ment triangle cambridge amsthdoc
    cambridgemods baseline-skip author authors cambridgestyle en-vir-on-ment polar}

  \setcounter{tocdepth}{2}% the toc normally lists sections;
% for the purposes of this document, this has been extended to subsections

%%%%%%%%%%%%%%%%%%%%%%%%%%%%%%%%%%%%%



  \begin{document}


%--------Meta Data: Fill in your info------
\title[ \ \\ \ \\ \ \\Translated by\\ \ \\
F A M Frescura \\Centre for Theoretical Physics
\\University of the Witwatersrand]
{Elie Cartan \\ \ \\ {\Large 1934-1935 Lectures on} \\ \ \\ The Theory of Spaces\\[.4cm] with Projective Connection\\ \ \\Part II --- Projectively Connected Spaces \\ \ \\ \Large Published by Gauthier-Villars, Paris, 1937}




%\date{DEADLINE AUGUST 26, 2011}

%%%%%%%%%%%%%%%%%%%%%%%%%%%%%%%%%%%%%


\maketitle
\frontmatter%%%%%%%%%%%%%%%%%%%%%%%%%%%%%%%%%%%%%%%%%%%%%%%%%%%%%%

%\include{dedic}
%\include{foreword}
%\include{acknow}

\tableofcontents

%\include{acronym}


\mainmatter%%%%%%%%%%%%%%%%%%%%%%%%%%%%%%%%%%%%%%%%%%%%%%%%%%  Part I



%============================================================    


%\part{Projective Differential Geometry}

% !TEX root = [Cartan]-ProjConnection.tex

\chapter{The Concept of a Space with Projective Connection}
             
\section{Geometric introduction of the concept of a space with projective connection. Two dimensional manifolds.}

% !TEX root = [Cartan]-ProjConnection.tex

%  Section completed 12 Jul 2017

{\bf 1.  Establishment of a projective connection on a surface. --- }
%
The concept of a space with projective connection can be related to the Gauss problem. Consider, in ordinary Euclidean space, an arbitrary surface $S$ having a given $ds^2$, and consider all the properties of the surface that depend only on $ds^2$. The set of these properties constitutes a geometry of the surface, belonging also to all surfaces equipped with the same $ds^2$ (applicable on $S$). Note that these surfaces need not be concretely defined; the common $ds^2$ defines them {\em in abstracto}.

Consider now any two-dimensional manifold $V$, with or without concrete representation in ordinary space, and assign to it the above $ds^2$. The concepts arising from this $ds^2$, which have a well defined meaning on $S$ (and on surfaces applicable to $S$), have no meaning on $V$.

If we {\em agree} to attribute the same meaning, on $V$ and on $S$, to the same properties arising from $ds^2$, we can assimilte $V$ to $S$, at least insofar as these properties are concerned. This is the fundamental idea of Riemannian geometry; we will extend it to the projective domain.

The generalisation can be made in a variety of ways. Taking up here the idea of Gauss, consider in ordinary space any surface $S$ and any curve $C$ drawn on $S$. At each of its points, $C$ has a certain geodesic curvature. Imagine the developable surface $\Delta$ circumscribed to $S$ along $C$; it is well known that if $\Delta$ is developed onto a plane $\Pi$, the geodesic curvature of $C$ is preserved; the plane curve $\Gamma$,  development of $C$, corresponds isometrically to $C$ with equal geodesic curvatures at homologous points. If we consider only properties relating to length and geodesic curvature, we may consider the set of curves on the surface $S$ as a set of plane curves, and $S$ may, to a certain extent, be assimilated to a plane.

Geometrically, the operation which allows $C$ to be developed on a plane consists of a series of infinitesimal rotations around successive generators of the developable $\Delta$. Each rotation realises an isometric point correspondence between the planes tangent to $S$ at two infinitely close points on $C$ and can, up to infinitesimals of second order, be interpreted as an orthogonal projection of the points of one of the tangent planes onto the other. The plane transform $\Gamma$ of $C$ is thus the result of a series of successive orthogonal projections. 

The previous geometric construction suggests the following very simple generalisation, of a projective nature. 

Let $S$ be any surface; at each point $\bm A$ of $S$ attach, in accordance with a given law, a point $\bm B$ not situated on the tangent plane at $\bm A$ ($\bm B$ may be, for example, the vertex $A_3$ of the intrinsic frame with origin $\bm A$). Consider then the sequence of planes tangent to $S$ along any one of its curves $C$. If $\bm M$ is any point of $C$, and $\bm P$ is the point which the chosen law associates with it, project, from $\bm P$, the tangent plane at $\bm M$ onto the tangent plane at the infinitely close point $\bm M'$. By this process, we establish a homographic transformation between the points of any two consecutive tangent planes (and consequently also between any two non-consecutive tangent planes). The set of points homologous to the same arbitrary point $\bm H$ of the tangent plane at $\bm M$ constitutes a curve $D$ whose tangent at $\bm H$ obviously passes through $\bm P$; the $\infty^2$ curves $D$, trajectories of the different points of the plane (which could be called the involutes of $S$ with respect to $C$), establish homographic correspondences between the different tangent planes.

Consider in particular the $\infty^1$ trajectories issuing from the points of $C$; each of them will cut at a certain point $m$ the tangent plane at any fixed point $\bm A$ of $S$; the set of points thus obtained is a certain plane curve $\Gamma$. At each of its points $m$, $\Gamma$ has a certain projective arc (the origin of the arcs being fixed) and a certain projective curvature; {\em by convention, this arc and this projective curvature will be the arc and projective curvature of $C$ at point $\bm M$ homologous to $m$.}

It is clear that the choice of the point $\bm A$ does not influence the above definitions: replacing $\bm A$ by another point $\bm A'$ of $C$ amounts to  replacing $\Gamma$ by a projectively equal curve.

The construction which has just been presented  attributes to every curve drawn on surface $S$ an arc and a projective curvature; everything, if we consider only properties with respect to these two elements, is as if the different curves of $S$ were situated on a projective plane. The surface $S$ can then be assimilated to a plane, and we will say that it has {\em projective connection}.

The extension of the concept of projective connection to surfaces and to manifolds in $n$-dimensional space offers no difficulties.












% section complete








































% !TEX root = [Cartan]-ProjConnection.tex

%  Section completed 24 Aug 2017

{\bf 2. Analytical study of the properties of curves on a projectively connected surface. --- }
% 
We now propose to investigate analytically the properties (arc, projective curvature, geodesics) of curves on projectively connected surfaces, defined geometrically in the preceding paragraph.

Attach to each analytic point $\bm A$ of a surface $S$ a first-order projective frame $[\bm{AA}_1 \bm A_2 \bm A_3]$, where $\bm A_3$ is the point associated with point $\bm A$ (point $\bm B$ of the preceding paragraph) when establishing the projective connection on $S$.

The passage from a frame to an infinitely close frame is translated by the formulas  
\begin{eqnarray*}
d\bm A &=& \omega^0_0 \bm A + \omega^1_0 \bm A_1 + \omega^2_0 \bm A_2\, , 
\\
d\bm A_1 &=& \omega^0_1 \bm A + \omega^1_1 \bm A_1 + \omega^2_1 \bm A_2 + \omega^3_1 \bm A_3\, , 
\\
d\bm A_2 &=& \omega^0_2 \bm A + \omega^1_2 \bm A_1 + \omega^2_2 \bm A_2 + \omega^3_2 \bm A_3\, , 
\\
d\bm A_3 &=& \omega^0_3 \bm A + \omega^1_3 \bm A_1 + \omega^2_3 \bm A_2 + \omega^3_3 \bm A_3\, ,
\end{eqnarray*}
where the $\omega^j_i$ are linear forms with respect to the differentials $du, dv$ of the two parameters which define the different points of the surface.

Suppose that $\bm A$ traces a curve $C$ on $S$; $u$ and $v$ are then functions of the same parameter $t$, and, along the curve, we have
\begin{eqnarray*}
p^j_i \ dt
\end{eqnarray*}
where the quantities $p^j_i$ are functions of $t$.

Let us try to define analytically the trajectories (defined geometrically in paragraph 1) of the various points on the tangent plane to $S$ at $A$.

Every point $\bm M$ of the tangent plane has an expression of the form
\begin{eqnarray*}
\bm M &=& \bm A + x \bm A_1 + y \bm A_2;
\end{eqnarray*}
$\bm M$ will trace out a trajectory if the tangent to the locus of $\bm M$ passes through $\bm A_3$,
that is, if
\begin{eqnarray*}
\frac{d \bm M}{dt} &=& \lambda \bm M + \mu \bm A_3 .
\end{eqnarray*}
If we take into account the expression for $\bm M$, we see that the above condition is linear in $\bm A, \bm A _1, \bm A_2 , \bm A_3$; it breaks down into four equations, between which it will suffice to eliminate $\lambda$ and $\mu$ to have the differential equations of the trajectories we seek. To remove $\mu$, let us neglect the term in $A_3$: we are left with
\begin{eqnarray*}
\lefteqn{
(p^0_0 + x p^0_1 + y p^0_2 ) \bm A + \left( \frac{d x}{dt} + p^1_0 + x p^1_1 + y p^1_2 \right) \bm A_1
 }\\ && \hspace*{2cm}
 + \left( \frac{d y}{dt} + p^2_0 + x p^2_1 + y p^2_2 \right) \bm A_2
 = \lambda( \bm A + x \bm A_1 + y \bm A_2 ) ,
\end{eqnarray*}
Elimiantion of $\lambda$ gives the two equations
\begin{eqnarray}
\left.
\begin{array}{lll}
\displaystyle \frac{dx}{dt} + p^1_0 +x p^1_1 + y p^1_2 - x^2 p^0_1 -  xy p^0_2 & =  & 0,   
\\ && \\
\displaystyle \frac{dy}{dt} + p^2_0 +x p^2_1 + y p^2_2 - xy p^0_1 -  y^2 p^0_2 & =  & 0, 
\end{array}
\right\}
\label{eq:1-1}
\end{eqnarray}
where we have, as always, replaced $p^j_i - p^0_0$ by $p^j_i$. Equations (\ref{eq:1-1}) are identical to those which express, in plane geometry,  than the point with moving coordinates $x, y$ [with respect to the frame $(\bm A \bm A_1, \bm A_2)$] is fixed. This result could have been anticipated. Let, in fact, $\bm A$ and $\bm A'$ be two infinitely close points on $C$, $R$ and $R'$ the corresponding frames, $\bm M$ and $\bm M'$ the points where the same trajectory cuts the tangent planes at $\bm  A$ and $\bm A'$. The set of all trajectories establishes a projective correspondence between the two tangent planes; denote by $\overline R$ the frame homologous to $R'$ in this correspondence; the coordinates of $\bm M'$ with respect to $R'$ are those of $\bm M$ with respect to $R$. $\bm M$ remains fixed when we pass from $R$ to $\overline R$, whence the existence of the formulas of the type (\ref{eq:1-1}). The identity of these formulas with the formulas (\ref{eq:1-1}) follows from the fact that, as is easy to show, the components $p^j_i\, dt\ (i,j=0,1,2)$ of the infinitesimal displacement taking $R$ to $\overline R$, are equal to the corresponding components of the displacement taking $R$ to $R'$. 

The infinitesimal projective correspondences which made to correspond, step by step, the planes tangent to $S$ along $C$, give finally a finite projective correspondence between the plane tangent at any point $\bm A$ of $C$ and the plane tangent at a fixed point $\bm A_0$. By this correspondence, the frame $[\bm A \bm A_1 \bm A_2]$ attached to the point $\bm A$ [in the plane tangent to $\bm A$] is transformed into a certain reference $[\bm B \bm B_1 \bm B_2]$, attached to the plane involute $\Gamma$ of $C$, situated in the tangent plane at $A_0$, and the quantities $p^j_i\, dt\ (i,j=0,1,2)$ for passing from a frame $[\bm A \bm A_1 \bm A_2]$ to the infinitely close frame on $C$, are equal to the corresponding quantities on $\Gamma$.

$\Gamma$ is projectively defined by the components $p^j_i\, dt$  of the infinitesimal displacement of the frame $[\bm B \bm B_1 \bm B_2]$ associated with each of its points $B$. It is sufficient to study the properties of $\Gamma$ to have those of $C$.

Since the study of the properties of $\Gamma$  depend, as we know (Part I, Chapter II), on the system
\begin{eqnarray}
\left.
\begin{array}{lll}
\displaystyle \frac{d\bm B}{dt} &=& p^1_0 \bm B_1 + p^2_0 \bm B_2\, ,
\\ && \\
\displaystyle \frac{d\bm B_1}{dt} &=& p^0_1 \bm B + p^1_1 \bm B_1 + p^2_1 \bm B_2 \hspace*{1cm} (p^0_0 = 0),\hspace*{1cm}
\\ && \\
\displaystyle \frac{d\bm B_2}{dt} &=& p^0_2 \bm B + p^1_2 \bm B_1 + p^2_2 \bm B_2\, 
\end{array}
\right\}
\label{eq:1-2}
\end{eqnarray}
we are led to this important remark:

In the study of the properties of the surface $S$, considered as projectively connected space of dimension two, {\em only those quantities $\omega^j_i$ of the matrix of components of the infinitesimal displacement of the moving frame are involved that are situated in the first three lines and the first three columns.}











% section complete








































% !TEX root = [Cartan]-ProjConnection.tex

%  Section completed 24 Aug 2017
\ \\

{\bf 3. Arc, curvature, geodesics. --- }
% 
Since the functions $p$ of $t$ appearing in formulas (\ref{eq:1-2}) are assumed known, we can deduce the differential equation of third order satisfied by $\bm B$; it is sufficient to put this equation into the form
\begin{eqnarray*}
\frac{d^3 \bm B}{dt^3}  + p_1 \frac{d^2 \bm B}{dt^2} + p_2 \frac{d \bm B}{dt} + p_3 \bm B &=& 0 
\end{eqnarray*}
by the method indicated in Chapter II of the first Part, to obtain the expressions for the arc and the projective curvature of $\Gamma$, that is, of $C$. 

Let us look for the curves $C$ of the surface $S$ which develop into lines: we will call these the {\em geodesics} of $S$. 

The point $\bm B$ will describe a line if $d^2\bm B$ is a linear combination of $d\bm B$ and $\bm B$. The geodesics of $S$ are thus the curves along which the quantities $\omega^j_i$ vary in such a way that 
\begin{eqnarray*}
d^2\bm B &=& \lambda d\bm B + \mu\bm B .
\end{eqnarray*}
By assuming that $\omega^0_0$, which is allowed since $\bm B$ displaces on a curve, and by suppressing zero subscript indices, we have
\begin{eqnarray*}
d\bm B &=& \omega^1 \bm B_1 + \omega^2 \bm B_2\, ,
\\ 
d\bm B_1 &=& \omega^0_1 \bm B + \omega^1_1 \bm B_1 + \omega^2_1 \bm B_2 \, ,  
\\
d\bm B_2 &=& \omega^0_2 \bm B + \omega^1_2 \bm B_1 + \omega^2_2 \bm B_2\, ,
\end{eqnarray*}
and the defining relation for geodesics becomes
\begin{eqnarray*}
d \left( \omega^1 \bm B_1 + \omega^2 \bm B_2 \right) 
&=& \lambda \left( \omega^1 \bm B_1 + \omega^2 \bm B_2 \right)  + \mu\bm B .
\end{eqnarray*}
This equation represents three, between which we must eliminate $\lambda$ and $\mu$. The elimination of $\mu$ is done by neglecting the terms in $\bm B$; we find then
\begin{eqnarray*}
\left( d\omega^1 + \omega^1 \omega^1_1 + \omega^2 \omega^1_2 \right) \bm B_1 
+ \left( d\omega^2 + \omega^1 \omega^2_1 + \omega^2 \omega^2_2 \right) \bm B_2
&=& \lambda \left( \omega^1 \bm B_1 + \omega^2 \bm B_2 \right) .
\end{eqnarray*}
On equating the coefficients of $\bm B_1$ and $\bm B_2$ on the two sides, and on eliminating $\lambda$ from the two equations obtained, we obtain the following differential equation for geodesics:
\begin{eqnarray}
\frac{d\omega^1 + \omega^1 \omega^1_1 + \omega^2 \omega^1_2 }{\omega^1} 
&=& \frac{d\omega^2 + \omega^1 \omega^2_1 + \omega^2 \omega^2_2 }{\omega^2} \, .
\label{eq:1-3}
\end{eqnarray}

In (\ref{eq:1-3}), the $\omega^j_1$ are linear forms in the differentials  $du$ and $dv$ of the two variables $u$ and $v$ that determine the different points of $S$. If we take $u$ as the independent variable [$v=f(u)$], we establish that the differential equation defining the geodesics of $S$ takes the form 
\begin{eqnarray*}
v'' + A(u,v) v'^3 + B(u,v) v'^2 + C(u,v) v' + D(u,v) &=& 0.
\end{eqnarray*}

We can give the geodesics of $S$ the following geometric definition: they are the curves such that their osculating plane at any point $\bm A$ passes through the point $\bm A_3$ associated with $\bm A$ in the definition of the projective connection. This property is geometrically evident; we have furthermore
\begin{eqnarray*}
\lefteqn{
\left| \bm A_3\ \bm A\ d\bm A\ \bm d^2A \right|
 } \\ &=& 
  \left| \bm A_3\, , \bm A\, , \omega^1 \bm A_1 + \omega^2 \bm A_2 \, , \left(d\omega^1 + \omega^1 \omega^1_1 + \omega^2 \omega^1_2 \right) \bm A_1 + \left( d\omega^2 + \omega^1 \omega^2_1 + \omega^2 \omega^2_2 \right) \bm A_2
 \right|\, ,
\end{eqnarray*}
and this last determinant is zero according to equation (\ref{eq:1-3}).

If we take $\bm A_3$ at infinity on the (ordinary) normal at $\bm A$, we recover the geodesic of ordinary geometry. 

































% section complete









































\section{General spaces with projective connection.}

% !TEX root = [Cartan]-ProjConnection.tex

%  Section completed 24 Aug 2017
\ \\

{\bf 4. Introduction of the connection. --- }
% 
To extend the concept of projective connection to a space with any number of dimensions, let us put aside the geometric considerations that guided us in the case of two dimensions, and retain only the conclusion to which we have been led.

Of the components of the table 
\begin{eqnarray*}
\left|
\begin{array}{cccc}
\omega^0_0  &  \omega^1_0 & \omega^2_0 & 0  \\
 \omega^0_1  &  \omega^1_1 & \omega^2_1 & \omega^3_1  \\
\omega^0_2  &  \omega^1_2 & \omega^2_2 & \omega^3_2  \\
\omega^0_3  &  \omega^1_3 & \omega^2_3 & \omega^3_3 
\end{array}
\right|
\end{eqnarray*}
that define the infinitesimal displacement of the frame attached to $S$, only those that are in the first three lines and the first three columns have played a role in establishing the projective connection on $S$. By means of these last components, we have been able to define a differential geometry of curves on $S$, by identifying it, by a  convention, with that of the curves on the projective plane.

But it is clear that complete identification, from a projective point of view. of $S$ with a plane cannot be realised only by knowing the $\omega^j_1\ (i,j = 0, i, 2)$. This is due to the fact that these quantities do not satisfy the structure conditions of the projective plane. The equations of structure for $\omega^1$ and $\omega^2$ are certainly satisfied, but the same is not true for the others. We have, for example,
\begin{eqnarray*}
(\omega^2_1)' &=& [\omega^0_1 \omega^2_0] + [\omega^1_1 \omega^2_1] +[\omega^2_1 \omega^2_2] +[\omega^3_1 \omega^2_3] , 
\end{eqnarray*}
and this relation cannot be a structure equation for the projective plane unless the last term of the right hand side is zero, which does  happen in general.

The preceding remarks, coupled with the observation that, from the moment that we confined ourselves to the consideration of curves, we could have assumed that $\omega^0_0 = 0$, lead us to define thus the spaces with projective connection of $n$ dimensions.\footnote{This definition is longer than that which has been given, in the case of two dimensions, in the preceding paragraphs.}

Consider a space of $n$ dimensions, whose various points are determined by a system of $n$ coordinates $(u^1, u^2, ..., u^n)$. Let us give arbitrarily $n(n+2)$ differential forms, linear in $du^1, du^2, ..., du^n$,
\begin{eqnarray*}
\omega^i_0, \ \ \ \ \omega^i_j, \ \ \ \ \omega^0_i \hspace*{1.5cm} (i,j = 1,2,...,n),
\end{eqnarray*}
and consider the matrix
\begin{eqnarray}
| \omega^j_i |,
\label{eq:1-4}
\end{eqnarray}
whose first element $\omega^0_0$ is zero.

Knowledge of this matrix allows us, as we shall see, to attribute to any curve of the space considered the same projective properties as a certain curve of the n-dimensional projective space. We will say that the matrix (\ref{eq:1-4}) defines, in the space, a projective connection.

It will be convenient to suppose that at each point $\bm A$ of the space considered is attached a frame with origin $\bm A$, where passage from the frame at a point to the frame at an infinitely near point is defined by the quantities $\omega^\beta_\alpha$. The frame at the point $\bm A$ will define a projective space, which we will call the projective space tangent at $\bm A$ to the space with projective connection considered.

























































% section complete








































% !TEX root = [Cartan]-ProjConnection.tex

%  Section completed 24 Aug 2017
\ \\

{\bf 5. Reconstruction of projective space. --- }
% 
To represent more concretely the abstract idea of a space with projective connection, let us return to the ordinary projective space of n-dimensions, and let us recall how space can be reconstructed from the components (assumed to satisfy the compatibility conditions) of the infinitesimal displacement of the frame attached to any of its points.

Let 
\begin{eqnarray*}
\omega^i_j \hspace*{1.5cm} (i,j = 0,1,2,...,n; \ \ \omega^0_0 = 0 )
\end{eqnarray*}
be these components; these are forms linear in the differentials of the $n$ coordinates $u^1, ..., u^n$.

Choose an origin point $\bm A_0$ with coordinates $(u^i)_0$, and fix {\em arbitrarily} the frame $R_0$ at the point $\bm A_0$. It is a matter of positioning, with respect to $R_0$, the point with coordinates $(u^i)$ and the associated frame $R$.

Imagine any continuous path $(\gamma)$ joining the point $(u^i)_0$ to the point $(u^i)$. Along this path, the components $\omega^i_j$ with have known expressions of the form 
\begin{eqnarray*}
\omega^i_j &=& p^i_j \, dt, 
\end{eqnarray*}
where $t$ is the parameter that defines the various points of $\gamma$.

The functions $p$ used here, as well as those that will be used in what follows, will be assumed analytic, but the results obtained persist if this condition is not satisfied. 

If $[\bm A \bm A_1 \bm A_2...\bm A_n]$ is the frame (unknown) at a running point on $\gamma$, we will have along this path
\begin{eqnarray}
\frac{d\bm A_i}{dt} = p_i^0 \bm A + p_i^1 \bm A_1 + p_i^2 \bm A_2 + \cdots + p_i^n \bm A_n 
\label{eq:1-5}  \\
\ [ \ i,j = 0,1,...,n; \ \ \bm A_0 = \bm A;\ \ p^0_0 = 0 ]. \hspace*{.4cm}
\nonumber
\end{eqnarray}
(\ref{eq:1-5}) is a system of linear differential equations, in which the coefficients are known functions of $t$. Let $t_0$ be the value of $t$ corresponding to the point $(u^i_0)$; for $t=t_0$, we know the initial values of the unknown $\bm A_i$ (because the initial frame is given);system (\ref{eq:1-5}) admits, as we know, a well defined solution $\bm A_i(t)$ that satisfies the initial conditions; if $t_1$ is the value of $t$ corresponding to the point $(u^i_1)$, the set of quantities $\bm A_i(t_1)$ determine at the same time the point $u^i$ and the associated frame. The result is independent of the path ($\gamma$) joining the points $(u^i_0)$ and $(u^i_1)$.

Thus, as soon as we fix the frame at a given point of space, we can position, relative to this reference point, all other points of the space as well as the associated frames; the space has been reconstructed.

If we change the initial frame $R_0$, we obtain another reconstruction of the space, which differs from the first only by a homography, and is consequently projectively equal to the first. The properties of a given figure are the same regardless of which  reconstruction of the space is considered.







































% section complete








































% !TEX root = [Cartan]-ProjConnection.tex

%  Section completed 26 Aug 2017
\ \\

{\bf 6. Development of a curve in a space with projective connection  onto ordinary projective space.. --- }
% 
Consider now any space with projective connection, defined by $n(n+2)$ {\em arbitrary} forms $\omega^\beta_\alpha$.

If, guided by the example of ordinary projective space, we try to construct a projective image of the space, by interpreting the quantities $\omega^\beta_\alpha$ as the components of the infinitesimal displacement of a projective frame attached to the generating point, we are faced with an impossibility, arising from the fact that the conditions for compatibility are not satisfied.

Representation on the projective space becomes possible if, instead of considering the entire space with projective connection, we consider only a curve.

Take curve $C$ of the projectively connected space; along this curve we will have 
\begin{eqnarray*}
u^i &=& \varphi^i(t) ,
\end{eqnarray*}
and the quantities $\omega^\beta_\alpha$ will be of the form $p^\beta_\alpha\, dt$, the $p^\beta_\alpha$ being known functions of $t$.

In ordinary projective space, the $n (n + 2)$ quantities $p^\beta_\alpha$ define, up to an homography, a one-parameter family of frames, for which the components of the infinitemal displacement are $p^\beta_\alpha\, dt$; this family is determined by the integration of the system
\begin{eqnarray*}
\frac{d\bm A_i}{dt} &=& p^j_i \bm A_j.
\end{eqnarray*}
The origin $\bm A$ of the general frame of the above family describes a certain curve $\Gamma$, which we may regard as the development, or the image, of the curve $C$ on the projective space. 

Each curve $C$ of the space with projective connection considered  has a development $\Gamma$ on the projective space, and, by definition, the properties of $C$ are those of $\Gamma$.






















































% section complete








































% !TEX root = [Cartan]-ProjConnection.tex

%  Section completed 26 Aug 2017
\ \\

{\bf 7. Equivalent spaces with projective connection. --- }
% 
Consider two spaces with projective connection defined, the first by the $n (n + 2)$ forms $\omega^\beta_\alpha$, the second by the $n (n +2)$ forms $\overline \omega^\beta_\alpha$, with the two systems of forms depending on the same variables $u^1, u^2, ..., u^n$. 

We will say that the two spaces are {\em equivalent} (or {\em applicable}) if the geometric properties which can be given to the curves of the first (by the method indicated in the preceding paragraph) are identical to the properties we can ascribe to corresponding curves in the second.

We ask the following question: {\em How can we modify the given $\omega^\beta_\alpha$ corresponding to a certain space with projective connection without modifying the properties of the various  curves of this space?} The space defined by the new forms $\overline \omega^\beta_\alpha$ will thus be equivalent to the first.

The first problem to solve in order to answer the above question is the following: how do we transform the quantities $\omega^\beta_\alpha$ of a displacement of the frame along a curve $C$, when we pass from one projective connection to an equivalent  projective connection?

We shall also be led to establish that the changes found for a particular curve are valid for {\em all the curves of space}, so that the problem initially posed will be found solved in all generality.

In ordinary projective space, consider a curve $\gamma$, image of a curve $C$ of the projectively connected space. The changes of the frames (the substitutions on the forms $\omega^\beta_\alpha$) that retain the properties of $C$ have as images in the projective space the different displacements that preserve the origin. It thus comes down to determining these last displacements.

The different points $\bm A$ of $\Gamma$ are the origins of a family of one parameter $(t)$ of frames defined by the values ??of the quantities $\omega^\beta_\alpha$ along the curve; let us substitute for the frame $[\bm A \bm A_1 \bm A_2 ... \bm A_n]$ a frame with the same origin $[\overline{\bm A} \overline{\bm A}_1 \overline{\bm A}_2 ... \overline{\bm A}_n]$. The new frame will be defined by equations of the form
\begin{eqnarray}
\left.
\begin{array}{lll}
\overline{\bm A}  & =  & \bm A  \\
\overline{\bm A}_1  & =  & a^0_1\bm A + a^1_1 \bm A_1 + \cdots + a^n_1 \bm A_n \\
 \cdots &   & \cdots  \\
\overline{\bm A}_n  & =  & a^0_n\bm A + a^1_n \bm A_1 + \cdots + a^n_n \bm A_n   , 
\end{array}
\right\}
\label{eq:1-6}
\end{eqnarray}
where the $a^j_i$ are {\em arbitrary} functions of $t$.

It is a matter of calculating the $\overline \omega^\beta_\alpha$ ??relative to the new frame, knowing the $\omega^\beta_\alpha$ and the $a^j_i$. The result of the calculation is capable of being into a very simple condensed form, by introducting certain matrices, and by regarding the theory of operations (especially multiplication) of the latter.

Put
\begin{eqnarray*}
(\bm A) = \left(
\begin{array}{c}
\bm A \\  \bm A_1 \\\bm A_2 \\ \vdots \\ \bm A_n 
\end{array}
\right), \hspace*{1.5cm}
(\, \overline{\bm A}\,) = \left(
\begin{array}{c}
\overline{\bm A} \\  \overline{\bm A}_1 \\ \overline{\bm A}_2 \\ \vdots \\ \overline{\bm A}_n 
\end{array}
\right), 
\\
(\, a\,) = \left(
\begin{array}{cccc}
1 &  0  & \cdots & 0 \\
a^0_1 &  a^1_1  & \cdots & a^n_1 \\
\vdots &    &  & \vdots \\
a^0_n &  a^1_n  & \cdots & a^n_n 
\end{array}
\right);  \hspace*{1.3cm}
\end{eqnarray*}
the first two matrices have only one column and $n+1$ rows; the last has $n+1$ rows and $n+1$ columns; multiplication by the first two is thus possible. 

From equations (\ref{eq:1-6}) we deduce immediately
\begin{eqnarray*}
(\, \overline{\bm A}\, ) &=& (a) (\bm A) .
\end{eqnarray*}
If we denote by $p^j_i\, dt$ the value of $\omega^j_i$ along $C$, and if we represent by $\left( \frac{d\bm A}{dt} \right)$ the matrix whose elements are the derivatives of the elements of $(\bm A)$ with respect to $t$, we have in this way
\begin{eqnarray*}
\left( \frac{d\bm A}{dt} \right) &=& 
\left(
\begin{array}{ccccc}
0 &  p^1  & p^2 & \cdots & p^n \\
p^0_1 &  p^1_1  & p^2_1  & \cdots & p^n_1 \\
\vdots &    &  & & \vdots \\
p^0_n &  p^1_n  & p^2_n  & \cdots & p^n_n 
\end{array}
\right)
\left(
\begin{array}{c}
\bm A \\  \bm A_1 \\\bm A_2 \\ \vdots \\ \bm A_n 
\end{array}
\right),
\end{eqnarray*}
or, denoting by $(p)$ the first matrix on the right hand side,
\begin{eqnarray*}
\left( \frac{d\bm A}{dt} \right) &=& (p) (\bm A) .
\end{eqnarray*}

Substitute for the frame $(\bm A)$ the frame $(\, \overline{\bm A}\, ) = (a) (\bm A)$, and calculate $\left( \frac{d\overline{\bm A}}{dt} \right)$. We have
\begin{eqnarray*}
\left( \frac{d\overline{\bm A}}{dt} \right) &=& \left(\frac{da}{dt} \right)) (\bm A) + (a) \left(\frac{d\bm A}{dt} \right) 
= \left(\frac{da}{dt} \right)) (\bm A) + (a) (p) \left(\bm A \right) .
\end{eqnarray*}

If we denote by $\left(a^{-1} \right)$ the matrix deduced from the array of the substitution inverse to (\ref{eq:1-6}), the matrix whose product by $(a)$ is the identity matrix
\begin{eqnarray*}
\left(
\begin{array}{cccc}
1 &  0   & \cdots & 0 \\
0 &  1    & \cdots & 0 \\
\vdots &    &  & \vdots \\
0 &  0    & \cdots & 1 
\end{array}
\right),
\end{eqnarray*}
and which we shall call the {\em inverse} of $(a)$, we have
\begin{eqnarray*}
(\bm A) &=& (a^{-1}) (\, \overline{\bm A}\, ) ,
\end{eqnarray*}
and the expression for $\left( \frac{d\overline{\bm A}}{dt} \right)$ can be written
\begin{eqnarray*}
\left( \frac{d\overline{\bm A}}{dt} \right) &=& 
\left( \left(\frac{da}{dt} \right)  + (a) (p) \right) (a^{-1}) \left(\, \overline{\bm A}\, \right) .
\end{eqnarray*}
We see that the matrix which replaces $(p)$, when we replace the basis $(\bm A)$ by the basis $(\, \overline{\bm A} \, )$, is
\begin{eqnarray*}
\left( \overline{p} \right) &=& 
\left( \left(\frac{da}{dt} \right)  + (a) (p) \right) (a^{-1})  .
\end{eqnarray*}
To obtain a result that applies to an arbitrary curve, it is sufficient to multiply the two sides of the above relation by $dt$; we thus obtain, by reintroducing the $\omega^j_i$ in place of the $p^j_i\, dt$, in such a way that nothing recalls the original curve $C$ considered.
\begin{eqnarray}
(\, \overline \omega \,) &=& \left( (da) +(a) (\omega) \right) (a^{-1}) .
\label{eq:1-7}
\end{eqnarray}

Definitively:

{\em The changes that can be made to the forms $\omega^\beta_\alpha$, without modifying the properties of the curves of the space with projective connection defined by these forms, are those which we obtain by substituting for the matrix $(\omega)$ the matrix $(\, \overline \omega \,)$ defined by Equation (\ref{eq:1-7}), where the elements of $(a)$ (other than those of the first line) are chosen in a completely arbitrary way.}


































































% section complete








































% !TEX root = [Cartan]-ProjConnection.tex

%  Section completed 26 Aug 2017
\ \\

{\bf 8. Effective determination of the elements of the transformed matrix. --- }
% 
The elements of the matrices $(a)$, $(da)$ and $\omega)$, appearing on the right hand side of relation (\ref{eq:1-7}), are known as soon as the matrix $(a)$ is chosen. To calculate the elements of $(\overline \omega)$, it is enough to know $(a^{-1})$.

Put
\begin{eqnarray*}
(\, a^{-1} \,) = \left(
\begin{array}{cccc}
b^0_0 &  b^1_0  & \cdots & b^n_0 \\
b^0_1 &  b^1_1  & \cdots & b^n_1 \\
\vdots &    &  & \vdots \\
b^0_n &  b^1_n  & \cdots & b^n_n 
\end{array}
\right); 
\end{eqnarray*}
we must have
\begin{eqnarray*}
\left(
\begin{array}{cccc}
1 &  0  & \cdots & 0 \\
a^0_1 &  a^1_1  & \cdots & a^n_1 \\
\vdots &    &  & \vdots \\
a^0_n &  a^1_n  & \cdots & a^n_n 
\end{array}
\right)
 \left(
\begin{array}{cccc}
b^0_0 &  b^1_0  & \cdots & b^n_0 \\
b^0_1 &  b^1_1  & \cdots & b^n_1 \\
\vdots &    &  & \vdots \\
b^0_n &  b^1_n  & \cdots & b^n_n 
\end{array}
\right)
&=& 
\left(
\begin{array}{cccc}
1 &  0   & \cdots & 0 \\
0 &  1    & \cdots & 0 \\
\vdots &    &  & \vdots \\
0 &  0    & \cdots & 1 
\end{array}
\right);
\end{eqnarray*}
this requires ($\lambda$ being a summation index)
\begin{eqnarray*}
a^\lambda_\alpha b^\beta_\lambda &=& 
\left\{
\begin{array}{lll}
1& \hspace*{1cm} & \mbox{if } \alpha = \beta , \\ 
0& \hspace*{1cm} & \mbox{if } \alpha \neq \beta .
\end{array}
\right.
\end{eqnarray*}

We deduce immediately from this that if $|a^\beta_\alpha |$ represents the determinant of the $a^\beta_\alpha$, the general element of the matrix $(a^{-1})$ is 
\begin{eqnarray*}
b^\beta_\alpha &=& \frac{\mbox{the minor with respect to the element } a^\beta_\alpha}{|a^\beta_\alpha |}\, .
\end{eqnarray*}
In particular, we have
\begin{eqnarray*}
b^0_0 = 1, \ \ \ \ b^i_0 = 0 \ \ (i\neq 0),
\end{eqnarray*}
so that $(a^{-1})$ has the same form as $(a)$:
\begin{eqnarray*}
(\, a^{-1} \,) = \left(
\begin{array}{cccc}
1 &  0  & \cdots & 0 \\
b^0_1 &  b^1_1  & \cdots & b^n_1 \\
\vdots &    &  & \vdots \\
b^0_n &  b^1_n  & \cdots & b^n_n 
\end{array}
\right); 
\end{eqnarray*}

We are now in a position to calculate the elements $\overline \omega^j_i$ of the matrix $(\, \overline \omega\, )$ the transform of $(\omega)$.

Note first that the fact that we assumed $\omega^0_0 = 0$ in $(\omega)$ does not lead to the nullity of the corresponding element $\overline \omega^0_0$ in $(\, \overline \omega\, )$. We have according to (\ref{eq:1-7})
\begin{eqnarray*}
\overline \omega^0_0 &=& da^\lambda_0 \, b^0_\lambda + a^\lambda_0 \omega^\mu_\lambda b^0_\mu ,
\end{eqnarray*}
where $\lambda$ and $\mu$ are summation indices. The first term on the right hand side is zero according to the form of $(a)$; the second reduces to $\omega^\mu_0 b^0_\mu \ (\mu = 0,1,2,...,n)$. The term in the sum $\omega^\mu_0 b^0_\mu$ with $\mu=0$ being zero, we will write by representing, here and in what follows, by Latin letters the indices that take values $1,2,...,n$ (excluding 0), 
\begin{eqnarray*}
\overline \omega^0_0 &=&   \omega^k_0 b^0_k .
\end{eqnarray*}

This is the new value of $\omega^0_0$; it is not zero in general, but we can set it to zero by subtracting $\omega^k_0 b^0_k$ from each of the terms on the principal diagonal of the matrix $(\, \overline \omega\, )$, which is equivalent to subtracting from this matrix the matrix
\begin{eqnarray*}
\omega^k_0 b^0_k (I),
\end{eqnarray*}
where $(I)$ represents the unit matrix.

We shall assume in what follows that $\overline \omega^0_0=0$.

The transforms of the other terms of the matrix $(\omega)$ are obtained similarly. We have, always according to (\ref{eq:1-7}),
\begin{eqnarray*}
\overline \omega^i_0 &=& da^\lambda_0 \, b^i_\lambda + a^\lambda_0 \omega^\mu_\lambda b^i_\mu ,
\end{eqnarray*}
so that, on introducing Latin indices and suppressing zero lower indices, 
\begin{eqnarray*}
\overline \omega^i_0 &=& b^i_k \omega^k .
\end{eqnarray*}
We find similarly, for $i,j \neq 0$,
\begin{eqnarray*}
\overline \omega^j_i &=& da^\lambda_i \, b^j_\lambda + a^\lambda_i \omega^\mu_\lambda b^j_\mu - b^0_k \omega^k \delta^j_i\, ,
\end{eqnarray*}
where $\delta^j_i$ has its usual meaning, which, according to the notational conventions made, can also be written
\begin{eqnarray*}
\overline \omega^j_i &=& da^k_i \, b^j_k + a^0_i \omega^k b^j_k + a^h_i \omega^k_h b^j_k - b^0_k \omega^k \delta^j_i\, .
\end{eqnarray*}
We have finally 
\begin{eqnarray*}
\overline \omega^0_i &=& da^0_i  + da^k_i \, b^0_k 
+ a^0_i \omega^k b^0_k 
+ a^k_i \omega^0_k 
+ a^k_i \omega^h_k b^0_h,
\end{eqnarray*}











































































% section complete








































% !TEX root = [Cartan]-ProjConnection.tex

%  Section completed 26 Aug 2017
\ \\

{\bf 9. The natural frame. --- }
% 
Consider any space with projective connection, defined by the  coordinates $(u^1, u^2, ..., u^n)$ of its various points and by the matrix $(\omega)$. We have seen that there exists an infinity of possible transformations for $(\omega)$, that preserve the properties of the various curves in the space. It is therefore natural to take advantage of this indeterminacy to make the matrix ($\omega$) as simple as possible.

First of all, let us turn our attention to the $\omega^i$. By one of these changes of frame previously defined, these quantities undergo linear transformations with arbitrary coefficients
\begin{eqnarray*}
\overline \omega^i &=& b^i_k \omega^k ;
\end{eqnarray*}
we can thus arrange it in such a way that $\omega^1, \omega^2, ...\omega^n$ take respectively the values $du^1, du^2, ..., du^n$:
\begin{eqnarray*}
 \omega^i &=& du^i\, ;
\end{eqnarray*}
the frames realising this condition will be said to be {\em semi-natural}.

The number of forms on which depend the frames attached to the surface is thus reduced to $n (n + 2) - n = n^2 + n$.

Consider the changes of frame that preserve the above form
of the $\omega^i$. For these changes, we will have
\begin{eqnarray*}
\overline \omega^i &=& \omega^i ,
\end{eqnarray*}
from which it follows that 
\begin{eqnarray*}
b^i_i = 1, \ \ \ \ b^j_i = 0\ \ (i\neq j).
\end{eqnarray*}
The matrix of the $b$ will thus have the form
\begin{eqnarray*}
(\, a^{-1} \,) = \left(
\begin{array}{ccccc}
1 & 0  & 0 & \cdots & 0 \\
b^0_1 &  1 & 0 & \cdots & 0 \\
b^0_2 &  0 & 1 & \cdots & 0 \\
\vdots &  &  &  & \vdots \\
b^0_n &  0  &  0  & \cdots & 1 
\end{array}
\right), 
\end{eqnarray*}
and the relations linking the elements of the two matrices $(a)$ and $(a^{-1})$ show that $(a)$ has the same form:
\begin{eqnarray*}
( a) = \left(
\begin{array}{ccccc}
1 & 0  & 0 & \cdots & 0 \\
a^0_1 &  1 & 0 & \cdots & 0 \\
a^0_2 &  0 & 1 & \cdots & 0 \\
\vdots &  &  &  & \vdots \\
a^0_n &  0  &  0  & \cdots & 1 
\end{array}
\right); 
\end{eqnarray*}
we establish immediately that 
\begin{eqnarray*}
(\, a^{-1} \,) = \left(
\begin{array}{ccccc}
1 & 0  & 0 & \cdots & 0 \\
- a^0_1 &  1 & 0 & \cdots & 0 \\
- a^0_2 &  0 & 1 & \cdots & 0 \\
\vdots &  &  &  & \vdots \\
- a^0_n &  0  &  0  & \cdots & 1 
\end{array}
\right). 
\end{eqnarray*}

Consider now the forms $\overline \omega^i_i$ obtained by taking into account the above specialisations. According to the general expression for the $\overline \omega^j_i$ given in paragraph 8, we have 
\begin{eqnarray*}
\overline \omega^1_1 &=& a^0_1\, du^1 + \omega^1_1 + a^0_k\, du^k
\\
\overline \omega^2_2 &=& a^0_2\, du^2 + \omega^2_2 + a^0_k\, du^k
\\
\vdots\ \   &=& \hspace*{1.6cm} \vdots
\\
\overline \omega^n_n &=& a^0_n\, du^n + \omega^n_n + a^0_k\, du^k.
\end{eqnarray*}
By adding these relations we obtain ($i$ being here a summation index)
\begin{eqnarray*}
\overline \omega^i_i &=&  \omega^i_i + (n+1) a^0_k\,, du^k\, :
\end{eqnarray*}
$\omega^i_i$ reproduces itself augmented by the linear form $(n+1) a^0_k\, du^k$; since this form is arbitrary, by the arbitrary character f the $a^0_k$, we can annihilate them. This operation reduces by one the number of arbitrary forms $\omega^\beta_\alpha$, which reduce finally to 
$$  n^2 + n -1. $$
After this new specialisation of the forms $\omega^\beta_\alpha$, on which depends the infinitesimal displacement of the frame, we will have 
\begin{eqnarray*}
(n+1) a^0_k\,, du^k &=& 0,
\end{eqnarray*}
from which 
\begin{eqnarray*}
a^0_k &=& 0,
\end{eqnarray*}
and the matric of the $a$ is reduced to the identity matrix
\begin{eqnarray*}
(a) &=& I .
\end{eqnarray*}

The most general projective connection that can be established on an $n$-dimensional continuum $(u^1, u^2, ..., u^n)$ depends, as we see, on being given $n^2+n-1$ arbitrary forms. Since each of these forms involves $n$ arbitrary functions of the coordinates $u^i$ (the coefficients of the differentials of the coordinates $u^i$), we can say that the most general projective connection that we can attribute to a
given continuum of $n$ dimensions depends on $n(n^2+n-1)$  arbitrary functions of coordinates.

The elements of the matrix associated with each connection are defined by formulas of the type
\begin{eqnarray}
\left.
\begin{array}{rcl}
 \omega^j_i & =  &  \Pi^j_{ik}\, du^k, \\
  &   &  \hspace*{2cm} (\Pi^{ik}_j = 0 ). \\
 \omega^0_i & =  &  \Pi^0_{ik}\, du^k,
\end{array}
\right\}
\label{eq:1-8}
\end{eqnarray}
Once the connection is chosen (the functions $\Pi$ have been fixed),  the motion of the frame attached to any point of the space is determined; the frame that formulas (8) attach abstractly to each  point of the space with projective connection considered is that which  we shall call the {\em natural frame} for that point.
























































































% section complete








































% !TEX root = [Cartan]-ProjConnection.tex

%  Section completed 26 Aug 2017
\ \\

{\bf 10. Infinitesimal change of the reference. --- }
% 
Move infinitely little around the origin the frame attached to each point of a space with projective connection, the displacement being one of those which leaves invariant the properties of the curves of the space. The matrix ($\omega$) oi the family of initial frames will undergo an infinitesimal variation ($\delta \omega$); we propose to find this variation.

Let ($a$) be the matrix that defines the transition from the frame attached to a point to the transformed frame with the same origin; this matrix must reduce to the unit matrix for a zero displacement [see folmulas (\ref{eq:1-6})]; we will have
\begin{eqnarray*}
(a) = (I) + (e),
\end{eqnarray*}
where $(e)$ has the form
\begin{eqnarray*}
(e) = \left(
\begin{array}{cccc}
0 & 0  &  \cdots & 0 \\
e^0_1 &  e^1_1  & \cdots & e^n_1 \\
e^0_2 &  e^1_2  & \cdots & e^n_2 \\
\vdots &  &    & \vdots \\
e^0_n &  e^1_n  & \cdots & e^n_n
\end{array}
\right), 
\end{eqnarray*}
where the non-zero elements $e$ are arbitrary linear forms of the differentials of the coordinates $u^i$ of the various points of the space.

We have here, by neglecting infinitesimals of second order, 
\begin{eqnarray*}
(a^{-1}) = (I) - (e),
\end{eqnarray*}
and consequently, according to equations (\ref{eq:1-7}) of paragraph 7, 
\begin{eqnarray*}
(\, \overline \omega\, ) = (de) + \left[ (I) + (e) \right] (\omega) \left[ (I) - (e) \right] .
\end{eqnarray*}

The first element of the matrix $(\, \overline \omega\, )$ is $- e^0_k \omega^k$; we will reduce it to zero by writing
\begin{eqnarray*}
(\, \overline \omega\, ) = (de) + \left[ (I) + (e) \right] (\omega) \left[ (I) - (e) \right] + e^0_k \omega^k (I).
\end{eqnarray*}

We see that 
\begin{eqnarray*}
(\, \overline \omega\, ) = (de) + (\omega) + (e) (\omega)  - (\omega) (e) + e^0_k \omega^k (I) ,
\end{eqnarray*}
and consequently, denoting by $(\delta \omega)$ the increment undergone by the matrix $\omega$ in the course of the displacement considered, 
\begin{eqnarray}
(\delta \omega) = (de)  + (e) (\omega)  - (\omega) (e) + e^0_k \omega^k (I)  .
\label{eq:1-9}
\end{eqnarray}
Explicitly, we obtain
\begin{eqnarray*}
\delta \omega^i &=& - \omega^k e^i_k , \\
\delta \omega^j_i &=& de^j_i + e^\lambda_i \omega^j_\lambda - \omega^\lambda_i e^j_\lambda + e^0_k \omega^k \delta^j_i , \\
\delta \omega^0_i &=& de^0_i + e^k_i \omega^0_k - \omega^k_i e^0_k .
\end{eqnarray*}

By means of the above formulas, we rover easily the results of the preceding paragraph on the existence of the natural frame.







































































































% section complete













































% completed



\ \\[3cm]
\begin{center}
\rule{2cm}{.05cm} \ \ $\circ$ \ \ \rule{2cm}{.05cm} 
\end{center} 






























% !TEX root = [Cartan]-ProjConnection.tex

\chapter{Differentiation of Spaces with Projective Connection and of Ordinary Projective Space. Cycles and Associated Displacements.}
             
\section{Introduction to the concept of a cycle}

%\input{PS-Section11}
%\input{PS-Section12}
%\input{PS-Section13}


\section{Study of a space with projective connection in the neighbourhood of one of its points}

%\input{PS-Section14}
%\input{PS-Section15}
%\input{PS-Section16}

%Chapter completed 25 Jun 2017































% !TEX root = [Cartan]-ProjConnection.tex

\chapter{Curvature and Torsion of a Space with Projective Connection}


\section{Concepts of the tensor calculus in projective geometry}
             
% !TEX root = [Cartan]-ProjConnection.tex

%  Section completed 28 Aug 2017
\ \\

{\bf 17. Tensors. --- }
% 
To present clearly the study of curvature and torsion of a projectively connected space, we need to present some introductory concepts from the tensor calculus in projective geometry.

Consider an entity of any kind (geometric, mechanical, physical, etc.), defined analytically, in a projective space of $n$ dimensions, by $r$ quantities $X_1, X_2,..., X_r$ that depend on the system of reference.

This entity is a {\em tensor} if the $r$ quantities $X_i$ which serve to define it in the chosen frame (its {\em components}) undergo a linear transformation for each change of coordinates that leaves fixed the origin $\bm A$ of the frame, where the coefficients of the transformation depend only on the quantities that define analytically this change of coordinates.

The coordinate transformations we consider are defined by formulas of the form
\begin{eqnarray}
x^i &=& \frac{a^i_k \overline x^k}{1 + a^0_k \overline x^k} \hspace*{1.5cm}
\mbox{($k$ is the summation index),}
\label{eq:3-1}
\end{eqnarray}
where the quantities $a^i_k, a^0_k\ (i, k = i, 2, ..., n)$ are arbitrary; If the $r$ numbers $X_1, X_2,..., X_r$ are the components of a tensor, each transformation of coordinates of type (\ref{eq:3-1}) will lead, for the components of the tensor, to transformations of the form 
\begin{eqnarray}
X_i &=& \alpha^k_i \overline X_k\, ,
\label{eq:3-2}
\end{eqnarray}
where the quantities $\alpha^k_i$ depend only on the $a^\beta_\alpha$.

Let us give $r$ variable quantities $X_1, X_2, ..., X_r$, devoid of any concrete meaning, and let us assume that, for any change of coordinates of the form (\ref{eq:3-1}),  these quantities undergo a well defined linear transformation (independent of the values of the variables) of the form (\ref{eq:3-2}). Under these conditions, can we {\em regard the $X_i$ as the components of a certain tensor?} In other words, is it the case that, to define a tensor, the transformation (\ref{eq:3-2}) corresponding to formulas (\ref{eq:3-1}) can be given arbitrarily?

Substitution (\ref{eq:3-2}) being assumed given, the components of the tensor, assumed to exist, will be known for all frames as soon as they are known for a given frame.

Let the $X_i$ be the components with respect to a given frame $R$; subject $R$ to a transformation of type (\ref{eq:3-1}), which we will denote by $T_1$; the $X_i$ will  undergo the corresponding transformation (\ref{eq:3-2}) (which we will denote by $S_1$), and will be transformed into $\overline X_i$. Subject the new frame to a new transformation $T_2$ of type (\ref{eq:3-1}); the $\overline X_i$ will undergo the corresponding transformation $S_2$, and will be transformed into $\overline{\overline X}_i$ The product $(T_2 T_1)$ is a certain transformation of the initial co-ordinates $x^i$  which, if the tensor actually exists, must lead for the components of the latter to the transformation $(S_2S_1)$ which transforms $X_i$ into $\overline{\overline X}_i$.

If (\ref{eq:3-2}) is chosen arbitrarily, the transformation $(T_2 T_1)$ on the $x$'s will not lead to the substitution $(S_2 S_1)$ on the $X$'s. The quantities $X_i$ can therefore not be regarded as the components of a tensor unless, to the product of two changes of variables, formulas (\ref{eq:3-2}) lead to the product of the two corresponding transformations of the components.

If the above condition is satisfied, we can say that the quantities $X_1, X_2,...,X_r$ define a tensor; but, if the existence of the tensor is guaranteed, its concrete definition might be very difficult to obtain.

We can cast the condition for existence of a tensor into the following form: 

For the variables $X_i$ to define the components of a tensor, it is necessary and sufficient that the substitutions $X_i = \alpha^k_i \overline X_k$ define a linear representation of the group of projective transformations that leave the origin fixed.






































































































% section complete








































% !TEX root = [Cartan]-ProjConnection.tex

%  Section completed 28 Aug 2017
\ \\

{\bf 18. Examples of tensors. Contravariant vectors. --- }
% 
Consider an analytic point $\bm M$, with homogeneous coordinates $y^0, y^1, ..., y^n$. Under the coordinate transformation (\ref{eq:3-1}), the $y$'s undergo the transformation
\begin{eqnarray*}
y^0 &=& \rho (\overline y^0 + a^0_k \overline y^k ), \\
y^\ell &=& \rho a^i_k \overline y^k ,
\end{eqnarray*}
where $\rho$ is an arbitrary factor.

Let us fix the above analytic transformation by taking $\rho$ equal to {\em one}; the coordinates of the analytic point $\bm M$ thus undergo, when we change the frame, a linear transformation that depends only on the initial frame and the final frame, and the set of transformations corresponding to different changes of frame form a group; the point $\bm M$ can thus be regarded as a tensor with components $y^0, y^1, ..., y^n$.

The origin $\bm A$ of coordinates and the analytic point $\bm M$ define an analytic vector (with origin $\bm A$ and tip $\bm M$), whose components $X^\alpha$ are the coordinates $y^0, y^1, ..., y^n$ of $\bm M$. This vector ({\em contravariant analytic vector}) is a tensor of the same titre as its tip $\bm M$.

If we consider only the last $n$ components $X^i = y^i$ of a contravariant analytic vector, we establish that these quantities also satisfy the definition of a tensor; we will designate this tensor under the name of {\em contravariant vector}.





















































































































% section complete








































% !TEX root = [Cartan]-ProjConnection.tex

%  Section completed 28 Aug 2017
\ \\

{\bf 19. Variations of a contravariant vector for an infinitesimal displacement of the frame that leaves the origin fixed. --- }
% 
Changes of coordinates of the form 
\begin{eqnarray}
\left. \begin{array}{rcl}
y^0 &=& \overline y^0 + a^0_k \overline y^k , \\
&&\hspace*{3cm} (i, k = 1,2,...,n) , \\
y^i &=&  a^i_k \overline y^k 
\end{array} \right\}
\label{eq:3-3}
\end{eqnarray}
which led us to the definitions of a contravariant analytic vector and a contravariant vector, correspond, as is easy to verify, to a displacement of frame defined by the matrix
\begin{eqnarray*}
(a) &=& \left(
\begin{array}{ccccc}
 1 & 0  & 0 & \cdots & 0 \\
 a^0_1 &  a^0_2 & \cdot & \cdots & a^n_1 \\
 \vdots & \vdots  &   &  & \vdots \\
 a^0_n &  a^1_n & \cdot & \cdots & a^n_n 
\end{array}
\right)
\end{eqnarray*}
 We have in fact, by expressing the point $\bm M$ in two different ways
 \begin{eqnarray*}
y^0 \bm A + y^i \bm A_i &=& \overline y^0 \, \overline{\bm A} + \overline y^i\, \overline{\bm A}_i\, , 
\end{eqnarray*}
where $(\overline{\bm A} \overline{\bm A}_1 ... \overline{\bm A}_n)$ is the displaced frame, or also, taking into account  (\ref{eq:3-3}), 
 \begin{eqnarray*}
(\overline y^0 + a^0_k \overline y^k )  \bm A + a^i_k \overline y^k  \bm A_i &=& \overline y^0 \, \overline{\bm A} + \overline y^i\, \overline{\bm A}_i\, , 
\end{eqnarray*}
we deduce the formulas
\begin{eqnarray*}
\overline{\bm A} &=& \bm A , \\
\overline{\bm A}_i &=& a^0_i \bm A + a^k_i \bm A_k,
\end{eqnarray*}
which indeed define the displacement of the frame indicated. 

We will have, in what follows, to consider infinitesimal coordinate transformations (infinitesimal displacements of the frame) that leave fixed the origin. The matrix that corresponds to one of these changes will be of the form (see \S 10)
\begin{eqnarray}
(a) &=& (I) + (e).
\label{eq:3-4}
\end{eqnarray}

If we denote by $\delta y^\alpha\ (\alpha=0,1,2,...,n)$ the variation undergone by the coordinate $y^\alpha$ of the fixed point $\bm M$ when we pass from the first frame to the second ($\overline y^\alpha = y^\alpha + \delta y^\alpha$), we can write, taking into account  (\ref{eq:3-3}) and expression (\ref{eq:3-4}) for $(a)$,
\begin{eqnarray*}
\delta y^0 + e^0_k y^k &=& 0, \\
\delta y^0i+ e^i_k y^k &=& 0 ; 
\end{eqnarray*}
pour the variation of the non-homogeneous coordinates, we find 
\begin{eqnarray*}
\delta x^i + e^i_k x^k - e^0_k x^i x^k &=& 0 . 
\end{eqnarray*}

We deduce immediately from these formulas those which define the infinitesimal variations of the components of the two tensors previously defined. For the analytic contravariant vector $(X^\alpha)$ with origin $\bm A$ and tip $\bm M$, we have 
\begin{eqnarray*}
\delta X^\alpha = - e^\alpha_k X^k,
\end{eqnarray*}
and for the contravariant vector (with $n$ components $X^i$)
\begin{eqnarray*}
\delta X^i = - e^i_k X^k .
\end{eqnarray*}














































































































% section complete








































% !TEX root = [Cartan]-ProjConnection.tex

%  Section completed 28 Aug 2017
\ \\

{\bf 20. Covariant vectors. --- }
% 
Considering a fixed point and a moving frame with fixed origin led us to the two tensors previously defined. The corresponding consideration of a fixed hyperplane and of a moving frame with fixed origin will allow us to introduce two new tensors. 

Consider to this end an hyperplane, defined with respect to a frame $(\bm A \bm A_1 ... \bm A_n)$ by the equation
\begin{eqnarray*}
u_0 y^0 + u_1 y^1 + u_2 y^2 + \cdots + u_n y^n &=& 0, 
\end{eqnarray*}
where $y^0, y^1, ..., y^n$ are homogeneous coordinates of a running point of the hyperplane. 

Change the frame keeping the origin fixed; the equation of the hyperplane (assumed fixed) transforms into 
\begin{eqnarray*}
\overline u_0 \overline y^0 + \overline u_1 \overline y^1 + \overline u_2 \overline y^2 + \cdots + \overline u_n \overline y^n &=& 0, 
\end{eqnarray*}
If, in the first equation, we replace the $y$'s by their expressions as functions of the $\overline y$'s [formulas (\ref{eq:3-3})], we must recover the second equation. By expressing this fact, we obtain the following formulas which express the $\overline u$'s in terms of the $u$'s:
\begin{eqnarray*}
\overline u_0 &=& u_0, \\
\overline u_i &=& a^0_i u_0 + a^k_i u_k\, .
\end{eqnarray*}

These formulas display the tensor $(X_\alpha)$ with components $u_\alpha$. We will say that this tensor is a {\em covariant analytic vector}. The presence of the term in $u_0$ in the second of the two formulas above shows that the $u_i$ do not form a tensor. The qualities of contravariance and of covariance of a vector are linked, as we see, to the role played by the indices in the elements $a^\beta_\alpha$ in the transformation formulas of the components.  We will distinguish the contravariant vectors from the covariant vectors by the position of the index defining their different components ({\em superscript} index for a contravariant vector, {\em subscript} for a covariant vector).

It is clear that the first component $u_0$ of the tensor which has just been defined forms a particular tensor with only one component; this component remains invariant for all changes of frame that leave invariant the origin;we will thus say that the tensor $u_0$ is a {\em scalar tensor}. 

The variations undergone by the components of the covariant analytic vector $(X_\alpha)$, for an infinitesimal displacement of the frame about the origin, are 
\begin{eqnarray*}
\delta X_0 &=& 0, \\
\delta X_i &=& e^\alpha_i X_\alpha\, .
\end{eqnarray*}

Consider now an hyperplane passing through the origin; we have $u_0=0$; the $n$ quantities $u_i$ undergo a linear transformation for all changes of coordinates that leave invariant the origin; they thus define a tensor with components $X_i = u_i$; we shall call this new tensor a {\em covariant vector}. 

An infinitesimal displacement of the frame around the origin transforms $(X_i)$ by the formulas
\begin{eqnarray*}
\delta X_i &=& e^k X_k\, .
\end{eqnarray*}

The four tensors which have just been defined will be called in what follows the {\em elementary tensors}. 

As regards these tensors, we have been led to following findings. Given an elementary tensor with a Greek superscript index $(X^\alpha)$, if we consider only the components with Latin values $(1, 2, ..., n)$, these components define a new tensor $(X^i)$.

If we consider an elementary tensor with a Greek subscript index $(X_\alpha)$, changing the Greek index into a Latin index does not give a new tensor, but the first component $X_0$ provides a scalar tensor.


























































































































% section complete










































\section{Tensor algebra}

% !TEX root = [Cartan]-ProjConnection.tex

%  Section completed 28 Aug 2017
%\ \\

{\bf 21. Tensor operations. --- }
% 
Consider two tensors of the same nature (for example two contravariant analytic vectors) which ahve the same number of components, $(X^\alpha), (Y^\alpha)$. The sum of components on the same line of the two vectors $(X^\alpha + Y^\alpha)$ are clearly the components of a new tensor (contravariant analytic vector in the example indicated); this tensor is the {\em sum of the two tensors considered}. The definition of the sum extends to multiple tensors of the same nature, that admit the same number of components.
\ \\[.2cm]

{\em Multiplication. --- } 
The definition of the multiplication applies to two (or more) tensors of any nature.

Consider any two tensors, with components respectively
\begin{eqnarray*}
a_1, a_2, ..., a_r, \\
b_1, b_2, ..., b_s. 
\end{eqnarray*}

Consider the $rs$ quantities $a_ib_j$; a change of coordinates that preserves the origin transforms $a_i$ and $b_j$ respectively into
\begin{eqnarray*}
\overline a_i &=& \lambda^k_i a_k,  \\
\overline b_j &=& \mu^h_j b_h,  
\end{eqnarray*}
and thus $a_i b_j$ into 
\begin{eqnarray*}
\overline a_i \overline b_j  &=& \lambda^k_i \mu^h_j a_k b_h.  
\end{eqnarray*}

The quantities $a_i b_j$ define, as we see, a tensor with $(rs)$ components; this tensor is {\em the product} of the two first ones. 
\ \\{.2cm}

{\em Contraction of indices. --- }
Consider two vectors, one contravariant, the other covariant, $X^i$ and $Y_j$, referred to the same frame. The product of these two tensors is the tensor $X^iY_j$. It is easy to see that from this last tensor we can deduce a {\em scalar tensor} with components $X^iY_i$ ($i$ being a summation index).

We have in fact, for an infinitesimal displacement of the frame around the origin (see \S 19 and 20),
\begin{eqnarray*}
\delta(X^i Y_i) = Y_i\, \delta X^i + X^i\, \delta Y_i = - e^i_k X^k Y_i + e^k_i Y_k X^i = 0.
\end{eqnarray*}
The quantity $X^i Y_i$ has a fixed numerical value, independent of the system of coordinates, and indeed defines a scalar tensor (scalar product of the two vectors considered).

When we pass from the initial product $X^iY_j$ to the associated scalar, we say that we have {\em contracted the indices}.

The product of two analytic vectors, one contravariant $(X^\alpha)$, the other covariant $(Y_\beta)$, leads similarly to the contracted tensor $(X^\alpha Y_\alpha)$.

Consider now the product $X_\alpha Y^i$ of an analytic covariant vector and a contravariant vector; it does not yield a contracted tensor; we have in fact
\begin{eqnarray*}
\delta(X^i Y_i) = Y_i\, \delta X^i + X^i\, \delta Y_i = Y^i ( e^0_i X_0 + e^k_i X_k ) - e^i_k Y^k X_i = e^0_i X_0 Y^i ,
\end{eqnarray*}
and this last quantity is not zero. 

For the tensor $X_i Y^\alpha$, on the contrary, the contraction of indices is possible. The only components of $Y^\alpha$ that are involved in the composition of the sum $X_i Y^i$ are in effect the $Y^j$, and we saw above that the product $X_iY^j$ yields a contracted tensor. 

In summary, the contraction of indices for the product of two elementary tensors formed by two vectors, one contravariant, the other covariant, is possible if the two vectors are or are not simultaneously analytic (if the indices are all both Greek or both all Latin); in the contrary case, the contraction is possible or not according as the subscript index is a Latin index or a Greek index.

Starting from the elementary tensors, and having regard to the rules of multiplication and of contraction of indices, we can obtain as many new tensors as we want. 

Thus, multiplication of
\begin{eqnarray*}
X_i, \ \ \ Y_\alpha, \ \ \ Z^\beta, \ \ \ T^\gamma, \ \ \ U^k, \ \ \ V_\ell, 
\end{eqnarray*}
gives a new tensor with 
 $$ n(n+1)(n+1)(n+1)n.n $$
components, which we will represent by the symbol
$$  a^{\centerdot \, \centerdot \, \beta \gamma k \centerdot }_{i \alpha\, \centerdot \, \centerdot \, \centerdot \, \ell}\, , $$
where the dots indicate that a superscript index and a subscript  index are not stacked.

A tensor of the above form being assumed known, we will obtain new tensors from it by the following operations:
\begin{itemize}
\item replacement of a Greek superscript index by a Latin index;
\item replacement of a Greek subscript index by zero;
\item 
contraction of two indices, one superscript, the other subscript, both Greek or both Latin;
\item 
contraction of a Latin subscript index with a Greek superscript index.

\end{itemize}


































































































































% section complete








































% !TEX root = [Cartan]-ProjConnection.tex

%  Section completed 28 Aug 2017
\ \\

{\bf 22. Infinitesimal projective transformations considered as tensors. --- }
% 
Considering infinitesimal projective transformations will allow us to recover the tensors defined previously and will lead us to new tensors. different from the preceding ones in the sense that they can not be deduced from them by multiplicative means.

Consider a point $\bm A$ of the space, to which is associated frame  $R$ (with origin $\bm A$); attach to the point $\bm A$ a definite infinitesimal projective transformation; by this transformation, the point $\bm A$ is transformed into an infinitely close point $\bm A'$, and the frame $R$ into an infinitely close frame $R'$; the components (relative to $R$) of the projective transformation considered are those of the projective displacement taking $R$ onto $R'$. These components $\omega^\beta_\alpha$ are $n(n+2)$ in  number (we assume $\omega^0_0 = 0$).

When we implement a change of frame $R$ that preserves the origin $\bm A$, each of them undergoes a linear transformation that depends only on the frame $R$ and the transformed frame. The $n (n + 2)$ quantities $\omega^\beta_\alpha$ consequently form a tensor. The number of components of this tensor shows that it cannot  be deduced by multiplication of the tensors previously defined.
\ \\[.2cm]

{\em Variations of the components of the tensor $\omega^\beta_\alpha$ for an infinitesimal displacement of the frame  around the origin. --- }
%
Let us displace infinitesimally the frame $R$ around the origin $\bm A$; let $\overline R$ be its new position; denote by $e^\alpha_i \ (e^\alpha_0 = 0)$ the quantities that define the transition from $R$  to $\overline R$.

The projective infinitesimal transformation considered is a geometric transformation operating on the entire space. Applied to $\bm A$ and $R$ it gives $\bm A'$ and $R'$; applied to $\overline R$ it will give a certain frame $\overline R\,'$ with origin $\bm A'$. 
%
Since the set of two frames $[R', \overline R']$ is deduced by an infinitesimal projective displacement of the set $[R, \overline R]$, we can say that the components of the displacement taking $R'$ onto $\overline R\,'$, are those of the displacement taking $R$ to $\overline R$, that is to say $e\alpha_i$. 
%
The quantities $e^\alpha_i$ are thus not modified by the infinitesimal transformation considered, and if we denote by $d$ the symbol of variation corresponding to this transformation, def = 0 to be the symbol of variation corresponding to this transformation, we can write
\begin{eqnarray*}
d e^\alpha_i = 0.
\end{eqnarray*}
If $\delta$ is the symbol of the variation corresponding to the infinitesimal displacement of the frame $R$ around the origin, the fundamental formulas of structure constructed with the symbols $d$ and $\delta$ (see the first Part, \S 76) now give the infinitesimal variations $\delta \omega^\beta_\alpha$ of the components of the tensor previously introduced.

We have generally
\begin{eqnarray*}
\lefteqn{
\delta \omega^\beta_\alpha (d) - d \omega^\beta_\alpha (\delta)
= \omega^\lambda_\alpha (\delta) \omega^\beta_\lambda (d) 
- \omega^\lambda_\alpha (d) \omega^\beta_\lambda (\delta)
 } \\ && \hspace*{4cm}
 - \delta^\beta_\alpha \left[ \omega^k_0 (\delta) \omega^0_k (d) 
- \omega^k_0 (d) \omega^0_k (\delta) \right] ;
\end{eqnarray*}
here
\begin{eqnarray*}
\omega^\beta_\alpha (d) = \omega^\beta_\alpha, \ \ \ \ 
\omega^\beta_\alpha (\delta) = e^\beta_\alpha, \ \ \ \ 
e^\alpha_0  = 0 ,
 \end{eqnarray*}
from which the formulas we seek
\begin{eqnarray}
\delta \omega^\beta_\alpha = e^\lambda_\alpha \omega^\beta_\lambda - e^\beta_\lambda \omega^\lambda_\alpha + \delta^\beta_\alpha e^0_k \omega^k_0 .
\label{eq:3-5}
\end{eqnarray}































































































































% section complete








































% !TEX root = [Cartan]-ProjConnection.tex

%  Section completed 28 Aug 2017
\ \\

{\bf 23. Tensors related to the tensor $\omega^\beta_\alpha$. --- }
% 
{\em New definition of contravariant and covariant vectors. ---}
The above formulas display other tensors related to the tensor $\omega^\beta_\alpha$. We have
\begin{eqnarray*}
\delta \omega^i_0 = - e^i_k \omega^k_0 ;
\end{eqnarray*}
the quantities $\omega^i_0$ thus behave (see \S 19) like the components of a contravariant vector, and provide a tensor with $n$ components, which can be regarded as representing the set of all the infinitesimal projective transformations giving the origin the same displacement $(\omega^1_0, \omega^2_0, ..., \omega^n_0)$. Every infinitsimal contravariant vector can be considered from this last point of view. 

We can also deduce from formulas (\ref{eq:3-5}) 
\begin{eqnarray}
\delta \omega^j_i = e^0_i \omega^j_0 + e^k_i  \omega^j_k - e^j_k  \omega^k_i + \delta^j_i e^0_k \omega^k_0 .
\label{eq:3-6}
\end{eqnarray}
The presence of the terms in $\omega^j_0$ and $\omega^k_0$ on the right hand side of (\ref{eq:3-6}) show that the $\omega^j_i$ do not form a tensor; but the set of the $\omega^i_0$ and the $\omega^j_i$ form one, whose components are the $\omega^i_\alpha$. 

We have, finally, 
\begin{eqnarray*}
\delta \omega^0_i =  e^k_i \omega^0_k - e^0_k \omega^k_i \, ,
\end{eqnarray*}
which shows that the $\omega^0_i$ do not form a tensor. 

The tensor $\omega^\beta_\alpha$ which we have just studied is for us the first example of a tensor schematically represented by a letter with indices superimposed. The rules of addition and multiplication presented in connection with elementary tensors can be extended to the tensors of the above type; in particular, multiplication of multiple tensors of this type yield a tensor represented by the symbol
  $$ a^{i \gamma j \cdot \cdot \cdot}_{\alpha \beta 0 \cdot\cdot\cdot} \, , $$
and the particular cases of formula (\ref{eq:3-5}) which we have just examined has already led us to say that, in a tensor of  the above general type, we can replace a superscript Greek index by a Latin index and a subscript Greek index by zero.

We can deduce other tensors from the tensor $\omega^\beta_\alpha$ by specialising the infinitesimal transformation which was our starting point.

If we consider only the infinitesimal transformations that leave the origin fixed ($\omega^i_0=0$), we see that these transformations can be regarded as tensors with $n (n + 1)$ components $\omega^\alpha_i$; these tensors behave moreover like the product $X_i Y ^\alpha$ of a contravariant analytic vector and a covariant vector. The formulas for the transformation of the components $\omega^\alpha_i$ of an infinitesimal displacement of the frame  around the origin are
\begin{eqnarray*}
\delta \omega^j_i &=& e^k_i \omega^j_k - e^j_k  \omega^k_i\, , \\
\delta \omega^0_i &=& e^k_i \omega0_k - e^0_k  \omega^k_i\, .
\end{eqnarray*}
The first of these two formulas show that the components $\omega^j_i$ form a tensor giving rise to the contracted tensor $\omega^i_i$. According to the second, the $\omega^0_i$ do not form a tensor, but if one considers infinitesimal transformations for which $\omega^j_i = 0$, we get a tensor with $n$ components $\omega^i_0$ for which we have 
\begin{eqnarray*}
\delta \omega^0_i &=& e^k_i \omega0_k \, ;
\end{eqnarray*}
we recover the covariant vector.































































































































% section complete








































% !TEX root = [Cartan]-ProjConnection.tex

%  Section completed 17 Sep 2017
\ \\

{\bf 24. Contravariant and covariant analytic vectors. --- }
% 
By specifying appropriately the infinitesimal projective transformation which is at the origin of the present study, we have been able to find the two elementary tensors formed by the contravariant and covariant vectors. The other two elementary tensors, namely the contravariant and covariant analytic vectors, can also be connected to this transformation.

Let us again take up formulas (\ref{eq:3-6}), considering the $\omega^1_0, \omega^2_0, ..., \omega^n_0$ as given quantities (the displacement of the origin of the frame is given). If we only consider only the $\omega^i_i$, we obtain ($i$ being a summation index)
\begin{eqnarray*}
\delta \omega^i_i &=& e^0_i \omega^i_0 + e^k_i \omega^i_k - e^i_k \omega^k_i + n e^0_k \omega^k_0, 
\end{eqnarray*}
which we can write, by a change of indices,
\begin{eqnarray*}
\delta \omega^i_i &=&  (n + 1) e^0_k \omega^k_0 ;
\end{eqnarray*}
we have furthermore
\begin{eqnarray*}
\delta \omega^i_0 &=&  - e^i_k \omega^k_0  . 
\end{eqnarray*}

If we put 
\begin{eqnarray*}
- \frac{ \omega^i_i }{n+1} = X^0, \ \ \ \ \  \omega^k_0 = X^i, 
\end{eqnarray*}
we can write 
\begin{eqnarray*}
\delta X^\alpha &=& - e^\alpha_k X^k ,
\end{eqnarray*}
and we see that the $X^\alpha$ are the components of an analytic contravariant vector, and the class of infinitesimal projective transformations which give to the origin a given displacement and to the quantity
\begin{eqnarray*}
- \frac{ 1}{n+1}(\omega^1_1 +\omega^2_2 + \cdots + \omega^n_n )   
\end{eqnarray*}
a given value.





















































































































































% section complete










































\section{Tensor analysis}

%\input{PC-Section25}
%\input{PC-Section26}
%\input{PC-Section27}
%\input{PC-Section28}
%\input{PC-Section29}


\section{Curvature and torsion tensors and internal properties of the space}

%\input{PC-Section30}
%\input{PC-Section31}
%\input{PC-Section32}
%\input{PC-Section33}
%\input{PC-Section34}


\ \\[.5cm]
\begin{center}
\rule{2cm}{.03cm} \ \ o \ \ \rule{2cm}{.03cm} 
\end{center} 


%\end{document}






























% !TEX root = [Cartan]-ProjConnection.tex

\chapter{The Bianchi Identities}


\section{Differential forms of second degree and the exterior calculus}
             
%\input{PS-Section35}
%\input{PS-Section36}
%\input{PS-Section37}


\section{The Bianchi identities}

%\input{PS-Section38}
%\input{PS-Section39}
%\input{PS-Section40}
%\input{PS-Section41}
%\input{PS-Section42}

\ \\[.5cm]
\begin{center}
\rule{2cm}{.03cm} \ \ o \ \ \rule{2cm}{.03cm} 
\end{center} 


%\end{document}






























% !TEX root = [Cartan]-ProjConnection.tex

\chapter{Differential Equations of Second Order Related to the Theory of Spaces with Projective Connection}


\section{Geometrisation of differential equations of the type
$$\frac{d^2 v}{du^2} = A \left( \frac{d v}{du} \right)^3 + 3 B \left( \frac{d v}{du} \right)^2 + 3 C \frac{d v}{du} + D, $$
where $A,B,C,D$ are functions of $u$ and $v$}
             
%\input{PS-Section43}
%\input{PS-Section44}
%\input{PS-Section45}
%\input{PS-Section46}


\section{Plane geodesic representation of surfaces}

%\input{PS-Section47}
%\input{PS-Section48}
%\input{PS-Section49}
%\input{PS-Section50}
%\input{PS-Section51}
%\input{PS-Section52}


\section{Geodesics of space with projective connection of dimension $n$}

%\input{PS-Section53}
%\input{PS-Section54}

\ \\[.5cm]
\begin{center}
\rule{2cm}{.03cm} \ \ o \ \ \rule{2cm}{.03cm} 
\end{center} 


%\end{document}






























% !TEX root = [Cartan]-ProjConnection.tex

\chapter{Differential Geometry of Surfaces Embedded in Spaces with Projective Connection of Dimension 3}


\section{Study of elements of second order}
             
%\input{PS-Section55}
%\input{PS-Section56}
%\input{PS-Section57}
%\input{PS-Section58}
%\input{PS-Section59}


\section{Study of elements of third order}

%\input{PS-Section60}
%\input{PS-Section61}
%\input{PS-Section62}
%\input{PS-Section63}




\ \\[.5cm]
\begin{center}
\rule{2cm}{.03cm} \ \ o \ \ \rule{2cm}{.03cm} 
\end{center} 


%\end{document}






























% !TEX root = [Cartan]-ProjConnection.tex

\chapter{Holonomy Group of a Space with Projective Connection}


\section{Definition and properties of an holonomy group}
             
%\input{PS-Section64}
%\input{PS-Section65}
%\input{PS-Section66}
%\input{PS-Section67}
%\input{PS-Section68}
%\input{PS-Section69}
%\input{PS-Section70}
%\input{PS-Section71}


\section{Spaces with projective connection admitting a given holonomy group}

%\input{PS-Section72}
%\input{PS-Section73}
%\input{PS-Section74}
%\input{PS-Section75}


\section{Normal spaces with projective connection}

%\input{PS-Section76}
%\input{PS-Section77}


\ \\[.5cm]
\begin{center}
\rule{2cm}{.03cm} \ \ o \ \ \rule{2cm}{.03cm} 
\end{center} 


%\end{document}































                      


\end{document}











%%%%%%%%%%%%%%%%%%%%%%%%%%%%%%%%%%%%%%%%%%%


\begin{figure}[h]
\begin{center}
\includegraphics[width=6cm]
{Section-1-7-Fig-1.jpg}
 \end{center}
 \vspace*{-.5cm}\caption{\small  \label{fig-1} }
\end{figure}

%----------------------------

 \begin{floatingbox}[h] \caption{Floating box title}

%\label{fb}
\end{floatingbox}

%----------------------------  boxed equations - single eqns

\begin{equation}
             \fbox{$
\ell_\textrm{c}(\alpha) = 2\alpha - \log \left( \mathrm{e}^\alpha
+ \cdots + \mathrm{e}^{4\alpha} \right)
  $}
\end{equation}

%------------------------  Boxed equations - my version

 \begin{center}
 \fbox{ \begin{minipage}[b]{8.7cm} {\bf Maxwell Equations - Original Form}
\begin{eqnarray*}
    \nabla \cdot \vD &=& \rho  \\
    \nabla \cdot \vB &=& 0      \\
    \nabla \times \vE &=& - \frac{\partial \vB}{\partial t}  \\
    \nabla \times \vH &=&  \vJ +  \frac{\partial \vD}{\partial t}
\end{eqnarray*}
\end{minipage} }
\end{center}

%-----------------------
%-------------------

Boxed Equations:

\begin{equation}
             \fbox{$
\ell_\textrm{c}(\alpha) = 2\alpha - \log \left( \mathrm{e}^\alpha
+ \cdots + \mathrm{e}^{4\alpha} \right)
  $}
\end{equation}

%--------------------

Text Boxes:

\begin{floatingbox}[h] \caption{Floating box title}
There are grounds for cautious optimism that we may now be near the end of the search for the ultimate laws of nature -- Stephen Hawking
\label{fb} \end{floatingbox}

