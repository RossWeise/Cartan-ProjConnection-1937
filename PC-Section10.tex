% !TEX root = [Cartan]-ProjConnection.tex

%  Section completed 26 Aug 2017
\ \\

{\bf 10. Infinitesimal change of the reference. --- }
% 
Move infinitely little around the origin the frame attached to each point of a space with projective connection, the displacement being one of those which leaves invariant the properties of the curves of the space. The matrix ($\omega$) oi the family of initial frames will undergo an infinitesimal variation ($\delta \omega$); we propose to find this variation.

Let ($a$) be the matrix that defines the transition from the frame attached to a point to the transformed frame with the same origin; this matrix must reduce to the unit matrix for a zero displacement [see folmulas (\ref{eq:1-6})]; we will have
\begin{eqnarray*}
(a) = (I) + (e),
\end{eqnarray*}
where $(e)$ has the form
\begin{eqnarray*}
(e) = \left(
\begin{array}{cccc}
0 & 0  &  \cdots & 0 \\
e^0_1 &  e^1_1  & \cdots & e^n_1 \\
e^0_2 &  e^1_2  & \cdots & e^n_2 \\
\vdots &  &    & \vdots \\
e^0_n &  e^1_n  & \cdots & e^n_n
\end{array}
\right), 
\end{eqnarray*}
where the non-zero elements $e$ are arbitrary linear forms of the differentials of the coordinates $u^i$ of the various points of the space.

We have here, by neglecting infinitesimals of second order, 
\begin{eqnarray*}
(a^{-1}) = (I) - (e),
\end{eqnarray*}
and consequently, according to equations (\ref{eq:1-7}) of paragraph 7, 
\begin{eqnarray*}
(\, \overline \omega\, ) = (de) + \left[ (I) + (e) \right] (\omega) \left[ (I) - (e) \right] .
\end{eqnarray*}

The first element of the matrix $(\, \overline \omega\, )$ is $- e^0_k \omega^k$; we will reduce it to zero by writing
\begin{eqnarray*}
(\, \overline \omega\, ) = (de) + \left[ (I) + (e) \right] (\omega) \left[ (I) - (e) \right] + e^0_k \omega^k (I).
\end{eqnarray*}

We see that 
\begin{eqnarray*}
(\, \overline \omega\, ) = (de) + (\omega) + (e) (\omega)  - (\omega) (e) + e^0_k \omega^k (I) ,
\end{eqnarray*}
and consequently, denoting by $(\delta \omega)$ the increment undergone by the matrix $\omega$ in the course of the displacement considered, 
\begin{eqnarray}
(\delta \omega) = (de)  + (e) (\omega)  - (\omega) (e) + e^0_k \omega^k (I)  .
\label{eq:1-9}
\end{eqnarray}
Explicitly, we obtain
\begin{eqnarray*}
\delta \omega^i &=& - \omega^k e^i_k , \\
\delta \omega^j_i &=& de^j_i + e^\lambda_i \omega^j_\lambda - \omega^\lambda_i e^j_\lambda + e^0_k \omega^k \delta^j_i , \\
\delta \omega^0_i &=& de^0_i + e^k_i \omega^0_k - \omega^k_i e^0_k .
\end{eqnarray*}

By means of the above formulas, we rover easily the results of the preceding paragraph on the existence of the natural frame.







































































































% section complete







































