% !TEX root = [Cartan]-ProjConnection.tex

%  Section completed 28 Aug 2017
\ \\

{\bf 19. Variations of a contravariant vector for an infinitesimal displacement of the frame that leaves the origin fixed. --- }
% 
Changes of coordinates of the form 
\begin{eqnarray}
\left. \begin{array}{rcl}
y^0 &=& \overline y^0 + a^0_k \overline y^k , \\
&&\hspace*{3cm} (i, k = 1,2,...,n) , \\
y^i &=&  a^i_k \overline y^k 
\end{array} \right\}
\label{eq:3-3}
\end{eqnarray}
which led us to the definitions of a contravariant analytic vector and a contravariant vector, correspond, as is easy to verify, to a displacement of frame defined by the matrix
\begin{eqnarray*}
(a) &=& \left(
\begin{array}{ccccc}
 1 & 0  & 0 & \cdots & 0 \\
 a^0_1 &  a^0_2 & \cdot & \cdots & a^n_1 \\
 \vdots & \vdots  &   &  & \vdots \\
 a^0_n &  a^1_n & \cdot & \cdots & a^n_n 
\end{array}
\right)
\end{eqnarray*}
 We have in fact, by expressing the point $\bm M$ in two different ways
 \begin{eqnarray*}
y^0 \bm A + y^i \bm A_i &=& \overline y^0 \, \overline{\bm A} + \overline y^i\, \overline{\bm A}_i\, , 
\end{eqnarray*}
where $(\overline{\bm A} \overline{\bm A}_1 ... \overline{\bm A}_n)$ is the displaced frame, or also, taking into account  (\ref{eq:3-3}), 
 \begin{eqnarray*}
(\overline y^0 + a^0_k \overline y^k )  \bm A + a^i_k \overline y^k  \bm A_i &=& \overline y^0 \, \overline{\bm A} + \overline y^i\, \overline{\bm A}_i\, , 
\end{eqnarray*}
we deduce the formulas
\begin{eqnarray*}
\overline{\bm A} &=& \bm A , \\
\overline{\bm A}_i &=& a^0_i \bm A + a^k_i \bm A_k,
\end{eqnarray*}
which indeed define the displacement of the frame indicated. 

We will have, in what follows, to consider infinitesimal coordinate transformations (infinitesimal displacements of the frame) that leave fixed the origin. The matrix that corresponds to one of these changes will be of the form (see \S 10)
\begin{eqnarray}
(a) &=& (I) + (e).
\label{eq:3-4}
\end{eqnarray}

If we denote by $\delta y^\alpha\ (\alpha=0,1,2,...,n)$ the variation undergone by the coordinate $y^\alpha$ of the fixed point $\bm M$ when we pass from the first frame to the second ($\overline y^\alpha = y^\alpha + \delta y^\alpha$), we can write, taking into account  (\ref{eq:3-3}) and expression (\ref{eq:3-4}) for $(a)$,
\begin{eqnarray*}
\delta y^0 + e^0_k y^k &=& 0, \\
\delta y^0i+ e^i_k y^k &=& 0 ; 
\end{eqnarray*}
pour the variation of the non-homogeneous coordinates, we find 
\begin{eqnarray*}
\delta x^i + e^i_k x^k - e^0_k x^i x^k &=& 0 . 
\end{eqnarray*}

We deduce immediately from these formulas those which define the infinitesimal variations of the components of the two tensors previously defined. For the analytic contravariant vector $(X^\alpha)$ with origin $\bm A$ and tip $\bm M$, we have 
\begin{eqnarray*}
\delta X^\alpha = - e^\alpha_k X^k,
\end{eqnarray*}
and for the contravariant vector (with $n$ components $X^i$)
\begin{eqnarray*}
\delta X^i = - e^i_k X^k .
\end{eqnarray*}














































































































% section complete







































