% !TEX root = [Cartan]-ProjConnection.tex

%  Section completed 26 Aug 2017
\ \\

{\bf 6. Development of a curve in a space with projective connection  onto ordinary projective space.. --- }
% 
Consider now any space with projective connection, defined by $n(n+2)$ {\em arbitrary} forms $\omega^\beta_\alpha$.

If, guided by the example of ordinary projective space, we try to construct a projective image of the space, by interpreting the quantities $\omega^\beta_\alpha$ as the components of the infinitesimal displacement of a projective frame attached to the generating point, we are faced with an impossibility, arising from the fact that the conditions for compatibility are not satisfied.

Representation on the projective space becomes possible if, instead of considering the entire space with projective connection, we consider only a curve.

Take curve $C$ of the projectively connected space; along this curve we will have 
\begin{eqnarray*}
u^i &=& \varphi^i(t) ,
\end{eqnarray*}
and the quantities $\omega^\beta_\alpha$ will be of the form $p^\beta_\alpha\, dt$, the $p^\beta_\alpha$ being known functions of $t$.

In ordinary projective space, the $n (n + 2)$ quantities $p^\beta_\alpha$ define, up to an homography, a one-parameter family of frames, for which the components of the infinitemal displacement are $p^\beta_\alpha\, dt$; this family is determined by the integration of the system
\begin{eqnarray*}
\frac{d\bm A_i}{dt} &=& p^j_i \bm A_j.
\end{eqnarray*}
The origin $\bm A$ of the general frame of the above family describes a certain curve $\Gamma$, which we may regard as the development, or the image, of the curve $C$ on the projective space. 

Each curve $C$ of the space with projective connection considered  has a development $\Gamma$ on the projective space, and, by definition, the properties of $C$ are those of $\Gamma$.






















































% section complete







































