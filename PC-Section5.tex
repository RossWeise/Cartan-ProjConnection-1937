% !TEX root = [Cartan]-ProjConnection.tex

%  Section completed 24 Aug 2017
\ \\

{\bf 5. Reconstruction of projective space. --- }
% 
To represent more concretely the abstract idea of a space with projective connection, let us return to the ordinary projective space of n-dimensions, and let us recall how space can be reconstructed from the components (assumed to satisfy the compatibility conditions) of the infinitesimal displacement of the frame attached to any of its points.

Let 
\begin{eqnarray*}
\omega^i_j \hspace*{1.5cm} (i,j = 0,1,2,...,n; \ \ \omega^0_0 = 0 )
\end{eqnarray*}
be these components; these are forms linear in the differentials of the $n$ coordinates $u^1, ..., u^n$.

Choose an origin point $\bm A_0$ with coordinates $(u^i)_0$, and fix {\em arbitrarily} the frame $R_0$ at the point $\bm A_0$. It is a matter of positioning, with respect to $R_0$, the point with coordinates $(u^i)$ and the associated frame $R$.

Imagine any continuous path $(\gamma)$ joining the point $(u^i)_0$ to the point $(u^i)$. Along this path, the components $\omega^i_j$ with have known expressions of the form 
\begin{eqnarray*}
\omega^i_j &=& p^i_j \, dt, 
\end{eqnarray*}
where $t$ is the parameter that defines the various points of $\gamma$.

The functions $p$ used here, as well as those that will be used in what follows, will be assumed analytic, but the results obtained persist if this condition is not satisfied. 

If $[\bm A \bm A_1 \bm A_2...\bm A_n]$ is the frame (unknown) at a running point on $\gamma$, we will have along this path
\begin{eqnarray}
\frac{d\bm A_i}{dt} = p_i^0 \bm A + p_i^1 \bm A_1 + p_i^2 \bm A_2 + \cdots + p_i^n \bm A_n 
\label{eq:1-5}  \\
\ [ \ i,j = 0,1,...,n; \ \ \bm A_0 = \bm A;\ \ p^0_0 = 0 ]. \hspace*{.4cm}
\nonumber
\end{eqnarray}
(\ref{eq:1-5}) is a system of linear differential equations, in which the coefficients are known functions of $t$. Let $t_0$ be the value of $t$ corresponding to the point $(u^i_0)$; for $t=t_0$, we know the initial values of the unknown $\bm A_i$ (because the initial frame is given);system (\ref{eq:1-5}) admits, as we know, a well defined solution $\bm A_i(t)$ that satisfies the initial conditions; if $t_1$ is the value of $t$ corresponding to the point $(u^i_1)$, the set of quantities $\bm A_i(t_1)$ determine at the same time the point $u^i$ and the associated frame. The result is independent of the path ($\gamma$) joining the points $(u^i_0)$ and $(u^i_1)$.

Thus, as soon as we fix the frame at a given point of space, we can position, relative to this reference point, all other points of the space as well as the associated frames; the space has been reconstructed.

If we change the initial frame $R_0$, we obtain another reconstruction of the space, which differs from the first only by a homography, and is consequently projectively equal to the first. The properties of a given figure are the same regardless of which  reconstruction of the space is considered.







































% section complete







































