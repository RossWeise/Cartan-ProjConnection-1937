% !TEX root = [Cartan]-ProjConnection.tex

%  Section completed 28 Aug 2017
\ \\

{\bf 17. Tensors. --- }
% 
To present clearly the study of curvature and torsion of a projectively connected space, we need to present some introductory concepts from the tensor calculus in projective geometry.

Consider an entity of any kind (geometric, mechanical, physical, etc.), defined analytically, in a projective space of $n$ dimensions, by $r$ quantities $X_1, X_2,..., X_r$ that depend on the system of reference.

This entity is a {\em tensor} if the $r$ quantities $X_i$ which serve to define it in the chosen frame (its {\em components}) undergo a linear transformation for each change of coordinates that leaves fixed the origin $\bm A$ of the frame, where the coefficients of the transformation depend only on the quantities that define analytically this change of coordinates.

The coordinate transformations we consider are defined by formulas of the form
\begin{eqnarray}
x^i &=& \frac{a^i_k \overline x^k}{1 + a^0_k \overline x^k} \hspace*{1.5cm}
\mbox{($k$ is the summation index),}
\label{eq:3-1}
\end{eqnarray}
where the quantities $a^i_k, a^0_k\ (i, k = i, 2, ..., n)$ are arbitrary; If the $r$ numbers $X_1, X_2,..., X_r$ are the components of a tensor, each transformation of coordinates of type (\ref{eq:3-1}) will lead, for the components of the tensor, to transformations of the form 
\begin{eqnarray}
X_i &=& \alpha^k_i \overline X_k\, ,
\label{eq:3-2}
\end{eqnarray}
where the quantities $\alpha^k_i$ depend only on the $a^\beta_\alpha$.

Let us give $r$ variable quantities $X_1, X_2, ..., X_r$, devoid of any concrete meaning, and let us assume that, for any change of coordinates of the form (\ref{eq:3-1}),  these quantities undergo a well defined linear transformation (independent of the values of the variables) of the form (\ref{eq:3-2}). Under these conditions, can we {\em regard the $X_i$ as the components of a certain tensor?} In other words, is it the case that, to define a tensor, the transformation (\ref{eq:3-2}) corresponding to formulas (\ref{eq:3-1}) can be given arbitrarily?

Substitution (\ref{eq:3-2}) being assumed given, the components of the tensor, assumed to exist, will be known for all frames as soon as they are known for a given frame.

Let the $X_i$ be the components with respect to a given frame $R$; subject $R$ to a transformation of type (\ref{eq:3-1}), which we will denote by $T_1$; the $X_i$ will  undergo the corresponding transformation (\ref{eq:3-2}) (which we will denote by $S_1$), and will be transformed into $\overline X_i$. Subject the new frame to a new transformation $T_2$ of type (\ref{eq:3-1}); the $\overline X_i$ will undergo the corresponding transformation $S_2$, and will be transformed into $\overline{\overline X}_i$ The product $(T_2 T_1)$ is a certain transformation of the initial co-ordinates $x^i$  which, if the tensor actually exists, must lead for the components of the latter to the transformation $(S_2S_1)$ which transforms $X_i$ into $\overline{\overline X}_i$.

If (\ref{eq:3-2}) is chosen arbitrarily, the transformation $(T_2 T_1)$ on the $x$'s will not lead to the substitution $(S_2 S_1)$ on the $X$'s. The quantities $X_i$ can therefore not be regarded as the components of a tensor unless, to the product of two changes of variables, formulas (\ref{eq:3-2}) lead to the product of the two corresponding transformations of the components.

If the above condition is satisfied, we can say that the quantities $X_1, X_2,...,X_r$ define a tensor; but, if the existence of the tensor is guaranteed, its concrete definition might be very difficult to obtain.

We can cast the condition for existence of a tensor into the following form: 

For the variables $X_i$ to define the components of a tensor, it is necessary and sufficient that the substitutions $X_i = \alpha^k_i \overline X_k$ define a linear representation of the group of projective transformations that leave the origin fixed.






































































































% section complete







































