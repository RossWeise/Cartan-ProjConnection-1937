% !TEX root = [Cartan]-ProjConnection.tex

%  Section completed 12 Jul 2017

{\bf 1.  Establishment of a projective connection on a surface. --- }
%
The concept of a space with projective connection can be related to the Gauss problem. Consider, in ordinary Euclidean space, an arbitrary surface $S$ having a given $ds^2$, and consider all the properties of the surface that depend only on $ds^2$. The set of these properties constitutes a geometry of the surface, belonging also to all surfaces equipped with the same $ds^2$ (applicable on $S$). Note that these surfaces need not be concretely defined; the common $ds^2$ defines them {\em in abstracto}.

Consider now any two-dimensional manifold $V$, with or without concrete representation in ordinary space, and assign to it the above $ds^2$. The concepts arising from this $ds^2$, which have a well defined meaning on $S$ (and on surfaces applicable to $S$), have no meaning on $V$.

If we {\em agree} to attribute the same meaning, on $V$ and on $S$, to the same properties arising from $ds^2$, we can assimilte $V$ to $S$, at least insofar as these properties are concerned. This is the fundamental idea of Riemannian geometry; we will extend it to the projective domain.

The generalisation can be made in a variety of ways. Taking up here the idea of Gauss, consider in ordinary space any surface $S$ and any curve $C$ drawn on $S$. At each of its points, $C$ has a certain geodesic curvature. Imagine the developable surface $\Delta$ circumscribed to $S$ along $C$; it is well known that if $\Delta$ is developed onto a plane $\Pi$, the geodesic curvature of $C$ is preserved; the plane curve $\Gamma$,  development of $C$, corresponds isometrically to $C$ with equal geodesic curvatures at homologous points. If we consider only properties relating to length and geodesic curvature, we may consider the set of curves on the surface $S$ as a set of plane curves, and $S$ may, to a certain extent, be assimilated to a plane.

Geometrically, the operation which allows $C$ to be developed on a plane consists of a series of infinitesimal rotations around successive generators of the developable $\Delta$. Each rotation realises an isometric point correspondence between the planes tangent to $S$ at two infinitely close points on $C$ and can, up to infinitesimals of second order, be interpreted as an orthogonal projection of the points of one of the tangent planes onto the other. The plane transform $\Gamma$ of $C$ is thus the result of a series of successive orthogonal projections. 

The previous geometric construction suggests the following very simple generalisation, of a projective nature. 

Let $S$ be any surface; at each point $\bm A$ of $S$ attach, in accordance with a given law, a point $\bm B$ not situated on the tangent plane at $\bm A$ ($\bm B$ may be, for example, the vertex $A_3$ of the intrinsic frame with origin $\bm A$). Consider then the sequence of planes tangent to $S$ along any one of its curves $C$. If $\bm M$ is any point of $C$, and $\bm P$ is the point which the chosen law associates with it, project, from $\bm P$, the tangent plane at $\bm M$ onto the tangent plane at the infinitely close point $\bm M'$. By this process, we establish a homographic transformation between the points of any two consecutive tangent planes (and consequently also between any two non-consecutive tangent planes). The set of points homologous to the same arbitrary point $\bm H$ of the tangent plane at $\bm M$ constitutes a curve $D$ whose tangent at $\bm H$ obviously passes through $\bm P$; the $\infty^2$ curves $D$, trajectories of the different points of the plane (which could be called the involutes of $S$ with respect to $C$), establish homographic correspondences between the different tangent planes.

Consider in particular the $\infty^1$ trajectories issuing from the points of $C$; each of them will cut at a certain point $m$ the tangent plane at any fixed point $\bm A$ of $S$; the set of points thus obtained is a certain plane curve $\Gamma$. At each of its points $m$, $\Gamma$ has a certain projective arc (the origin of the arcs being fixed) and a certain projective curvature; {\em by convention, this arc and this projective curvature will be the arc and projective curvature of $C$ at point $\bm M$ homologous to $m$.}

It is clear that the choice of the point $\bm A$ does not influence the above definitions: replacing $\bm A$ by another point $\bm A'$ of $C$ amounts to  replacing $\Gamma$ by a projectively equal curve.

The construction which has just been presented  attributes to every curve drawn on surface $S$ an arc and a projective curvature; everything, if we consider only properties with respect to these two elements, is as if the different curves of $S$ were situated on a projective plane. The surface $S$ can then be assimilated to a plane, and we will say that it has {\em projective connection}.

The extension of the concept of projective connection to surfaces and to manifolds in $n$-dimensional space offers no difficulties.












% section complete







































