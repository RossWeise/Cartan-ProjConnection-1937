% !TEX root = [Cartan]-ProjConnection.tex

%  Section completed 28 Aug 2017
\ \\

{\bf 18. Examples of tensors. Contravariant vectors. --- }
% 
Consider an analytic point $\bm M$, with homogeneous coordinates $y^0, y^1, ..., y^n$. Under the coordinate transformation (\ref{eq:3-1}), the $y$'s undergo the transformation
\begin{eqnarray*}
y^0 &=& \rho (\overline y^0 + a^0_k \overline y^k ), \\
y^\ell &=& \rho a^i_k \overline y^k ,
\end{eqnarray*}
where $\rho$ is an arbitrary factor.

Let us fix the above analytic transformation by taking $\rho$ equal to {\em one}; the coordinates of the analytic point $\bm M$ thus undergo, when we change the frame, a linear transformation that depends only on the initial frame and the final frame, and the set of transformations corresponding to different changes of frame form a group; the point $\bm M$ can thus be regarded as a tensor with components $y^0, y^1, ..., y^n$.

The origin $\bm A$ of coordinates and the analytic point $\bm M$ define an analytic vector (with origin $\bm A$ and tip $\bm M$), whose components $X^\alpha$ are the coordinates $y^0, y^1, ..., y^n$ of $\bm M$. This vector ({\em contravariant analytic vector}) is a tensor of the same titre as its tip $\bm M$.

If we consider only the last $n$ components $X^i = y^i$ of a contravariant analytic vector, we establish that these quantities also satisfy the definition of a tensor; we will designate this tensor under the name of {\em contravariant vector}.





















































































































% section complete







































