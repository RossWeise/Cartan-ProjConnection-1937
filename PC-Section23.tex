% !TEX root = [Cartan]-ProjConnection.tex

%  Section completed 28 Aug 2017
\ \\

{\bf 23. Tensors related to the tensor $\omega^\beta_\alpha$. --- }
% 
{\em New definition of contravariant and covariant vectors. ---}
The above formulas display other tensors related to the tensor $\omega^\beta_\alpha$. We have
\begin{eqnarray*}
\delta \omega^i_0 = - e^i_k \omega^k_0 ;
\end{eqnarray*}
the quantities $\omega^i_0$ thus behave (see \S 19) like the components of a contravariant vector, and provide a tensor with $n$ components, which can be regarded as representing the set of all the infinitesimal projective transformations giving the origin the same displacement $(\omega^1_0, \omega^2_0, ..., \omega^n_0)$. Every infinitsimal contravariant vector can be considered from this last point of view. 

We can also deduce from formulas (\ref{eq:3-5}) 
\begin{eqnarray}
\delta \omega^j_i = e^0_i \omega^j_0 + e^k_i  \omega^j_k - e^j_k  \omega^k_i + \delta^j_i e^0_k \omega^k_0 .
\label{eq:3-6}
\end{eqnarray}
The presence of the terms in $\omega^j_0$ and $\omega^k_0$ on the right hand side of (\ref{eq:3-6}) show that the $\omega^j_i$ do not form a tensor; but the set of the $\omega^i_0$ and the $\omega^j_i$ form one, whose components are the $\omega^i_\alpha$. 

We have, finally, 
\begin{eqnarray*}
\delta \omega^0_i =  e^k_i \omega^0_k - e^0_k \omega^k_i \, ,
\end{eqnarray*}
which shows that the $\omega^0_i$ do not form a tensor. 

The tensor $\omega^\beta_\alpha$ which we have just studied is for us the first example of a tensor schematically represented by a letter with indices superimposed. The rules of addition and multiplication presented in connection with elementary tensors can be extended to the tensors of the above type; in particular, multiplication of multiple tensors of this type yield a tensor represented by the symbol
  $$ a^{i \gamma j \cdot \cdot \cdot}_{\alpha \beta 0 \cdot\cdot\cdot} \, , $$
and the particular cases of formula (\ref{eq:3-5}) which we have just examined has already led us to say that, in a tensor of  the above general type, we can replace a superscript Greek index by a Latin index and a subscript Greek index by zero.

We can deduce other tensors from the tensor $\omega^\beta_\alpha$ by specialising the infinitesimal transformation which was our starting point.

If we consider only the infinitesimal transformations that leave the origin fixed ($\omega^i_0=0$), we see that these transformations can be regarded as tensors with $n (n + 1)$ components $\omega^\alpha_i$; these tensors behave moreover like the product $X_i Y ^\alpha$ of a contravariant analytic vector and a covariant vector. The formulas for the transformation of the components $\omega^\alpha_i$ of an infinitesimal displacement of the frame  around the origin are
\begin{eqnarray*}
\delta \omega^j_i &=& e^k_i \omega^j_k - e^j_k  \omega^k_i\, , \\
\delta \omega^0_i &=& e^k_i \omega0_k - e^0_k  \omega^k_i\, .
\end{eqnarray*}
The first of these two formulas show that the components $\omega^j_i$ form a tensor giving rise to the contracted tensor $\omega^i_i$. According to the second, the $\omega^0_i$ do not form a tensor, but if one considers infinitesimal transformations for which $\omega^j_i = 0$, we get a tensor with $n$ components $\omega^i_0$ for which we have 
\begin{eqnarray*}
\delta \omega^0_i &=& e^k_i \omega0_k \, ;
\end{eqnarray*}
we recover the covariant vector.































































































































% section complete







































