% !TEX root = [Cartan]-ProjConnection.tex

%  Section completed 28 Aug 2017
\ \\

{\bf 20. Covariant vectors. --- }
% 
Considering a fixed point and a moving frame with fixed origin led us to the two tensors previously defined. The corresponding consideration of a fixed hyperplane and of a moving frame with fixed origin will allow us to introduce two new tensors. 

Consider to this end an hyperplane, defined with respect to a frame $(\bm A \bm A_1 ... \bm A_n)$ by the equation
\begin{eqnarray*}
u_0 y^0 + u_1 y^1 + u_2 y^2 + \cdots + u_n y^n &=& 0, 
\end{eqnarray*}
where $y^0, y^1, ..., y^n$ are homogeneous coordinates of a running point of the hyperplane. 

Change the frame keeping the origin fixed; the equation of the hyperplane (assumed fixed) transforms into 
\begin{eqnarray*}
\overline u_0 \overline y^0 + \overline u_1 \overline y^1 + \overline u_2 \overline y^2 + \cdots + \overline u_n \overline y^n &=& 0, 
\end{eqnarray*}
If, in the first equation, we replace the $y$'s by their expressions as functions of the $\overline y$'s [formulas (\ref{eq:3-3})], we must recover the second equation. By expressing this fact, we obtain the following formulas which express the $\overline u$'s in terms of the $u$'s:
\begin{eqnarray*}
\overline u_0 &=& u_0, \\
\overline u_i &=& a^0_i u_0 + a^k_i u_k\, .
\end{eqnarray*}

These formulas display the tensor $(X_\alpha)$ with components $u_\alpha$. We will say that this tensor is a {\em covariant analytic vector}. The presence of the term in $u_0$ in the second of the two formulas above shows that the $u_i$ do not form a tensor. The qualities of contravariance and of covariance of a vector are linked, as we see, to the role played by the indices in the elements $a^\beta_\alpha$ in the transformation formulas of the components.  We will distinguish the contravariant vectors from the covariant vectors by the position of the index defining their different components ({\em superscript} index for a contravariant vector, {\em subscript} for a covariant vector).

It is clear that the first component $u_0$ of the tensor which has just been defined forms a particular tensor with only one component; this component remains invariant for all changes of frame that leave invariant the origin;we will thus say that the tensor $u_0$ is a {\em scalar tensor}. 

The variations undergone by the components of the covariant analytic vector $(X_\alpha)$, for an infinitesimal displacement of the frame about the origin, are 
\begin{eqnarray*}
\delta X_0 &=& 0, \\
\delta X_i &=& e^\alpha_i X_\alpha\, .
\end{eqnarray*}

Consider now an hyperplane passing through the origin; we have $u_0=0$; the $n$ quantities $u_i$ undergo a linear transformation for all changes of coordinates that leave invariant the origin; they thus define a tensor with components $X_i = u_i$; we shall call this new tensor a {\em covariant vector}. 

An infinitesimal displacement of the frame around the origin transforms $(X_i)$ by the formulas
\begin{eqnarray*}
\delta X_i &=& e^k X_k\, .
\end{eqnarray*}

The four tensors which have just been defined will be called in what follows the {\em elementary tensors}. 

As regards these tensors, we have been led to following findings. Given an elementary tensor with a Greek superscript index $(X^\alpha)$, if we consider only the components with Latin values $(1, 2, ..., n)$, these components define a new tensor $(X^i)$.

If we consider an elementary tensor with a Greek subscript index $(X_\alpha)$, changing the Greek index into a Latin index does not give a new tensor, but the first component $X_0$ provides a scalar tensor.


























































































































% section complete







































