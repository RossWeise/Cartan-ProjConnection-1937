% !TEX root = [Cartan]-ProjConnection.tex

%  Section completed 24 Aug 2017
\ \\

{\bf 4. Introduction of the connection. --- }
% 
To extend the concept of projective connection to a space with any number of dimensions, let us put aside the geometric considerations that guided us in the case of two dimensions, and retain only the conclusion to which we have been led.

Of the components of the table 
\begin{eqnarray*}
\left|
\begin{array}{cccc}
\omega^0_0  &  \omega^1_0 & \omega^2_0 & 0  \\
 \omega^0_1  &  \omega^1_1 & \omega^2_1 & \omega^3_1  \\
\omega^0_2  &  \omega^1_2 & \omega^2_2 & \omega^3_2  \\
\omega^0_3  &  \omega^1_3 & \omega^2_3 & \omega^3_3 
\end{array}
\right|
\end{eqnarray*}
that define the infinitesimal displacement of the frame attached to $S$, only those that are in the first three lines and the first three columns have played a role in establishing the projective connection on $S$. By means of these last components, we have been able to define a differential geometry of curves on $S$, by identifying it, by a  convention, with that of the curves on the projective plane.

But it is clear that complete identification, from a projective point of view. of $S$ with a plane cannot be realised only by knowing the $\omega^j_1\ (i,j = 0, i, 2)$. This is due to the fact that these quantities do not satisfy the structure conditions of the projective plane. The equations of structure for $\omega^1$ and $\omega^2$ are certainly satisfied, but the same is not true for the others. We have, for example,
\begin{eqnarray*}
(\omega^2_1)' &=& [\omega^0_1 \omega^2_0] + [\omega^1_1 \omega^2_1] +[\omega^2_1 \omega^2_2] +[\omega^3_1 \omega^2_3] , 
\end{eqnarray*}
and this relation cannot be a structure equation for the projective plane unless the last term of the right hand side is zero, which does  happen in general.

The preceding remarks, coupled with the observation that, from the moment that we confined ourselves to the consideration of curves, we could have assumed that $\omega^0_0 = 0$, lead us to define thus the spaces with projective connection of $n$ dimensions.\footnote{This definition is longer than that which has been given, in the case of two dimensions, in the preceding paragraphs.}

Consider a space of $n$ dimensions, whose various points are determined by a system of $n$ coordinates $(u^1, u^2, ..., u^n)$. Let us give arbitrarily $n(n+2)$ differential forms, linear in $du^1, du^2, ..., du^n$,
\begin{eqnarray*}
\omega^i_0, \ \ \ \ \omega^i_j, \ \ \ \ \omega^0_i \hspace*{1.5cm} (i,j = 1,2,...,n),
\end{eqnarray*}
and consider the matrix
\begin{eqnarray}
| \omega^j_i |,
\label{eq:1-4}
\end{eqnarray}
whose first element $\omega^0_0$ is zero.

Knowledge of this matrix allows us, as we shall see, to attribute to any curve of the space considered the same projective properties as a certain curve of the n-dimensional projective space. We will say that the matrix (\ref{eq:1-4}) defines, in the space, a projective connection.

It will be convenient to suppose that at each point $\bm A$ of the space considered is attached a frame with origin $\bm A$, where passage from the frame at a point to the frame at an infinitely near point is defined by the quantities $\omega^\beta_\alpha$. The frame at the point $\bm A$ will define a projective space, which we will call the projective space tangent at $\bm A$ to the space with projective connection considered.

























































% section complete







































