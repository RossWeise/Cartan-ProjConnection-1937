% !TEX root = [Cartan]-ProjConnection.tex

%  Section completed 28 Aug 2017
\ \\

{\bf 22. Infinitesimal projective transformations considered as tensors. --- }
% 
Considering infinitesimal projective transformations will allow us to recover the tensors defined previously and will lead us to new tensors. different from the preceding ones in the sense that they can not be deduced from them by multiplicative means.

Consider a point $\bm A$ of the space, to which is associated frame  $R$ (with origin $\bm A$); attach to the point $\bm A$ a definite infinitesimal projective transformation; by this transformation, the point $\bm A$ is transformed into an infinitely close point $\bm A'$, and the frame $R$ into an infinitely close frame $R'$; the components (relative to $R$) of the projective transformation considered are those of the projective displacement taking $R$ onto $R'$. These components $\omega^\beta_\alpha$ are $n(n+2)$ in  number (we assume $\omega^0_0 = 0$).

When we implement a change of frame $R$ that preserves the origin $\bm A$, each of them undergoes a linear transformation that depends only on the frame $R$ and the transformed frame. The $n (n + 2)$ quantities $\omega^\beta_\alpha$ consequently form a tensor. The number of components of this tensor shows that it cannot  be deduced by multiplication of the tensors previously defined.
\ \\[.2cm]

{\em Variations of the components of the tensor $\omega^\beta_\alpha$ for an infinitesimal displacement of the frame  around the origin. --- }
%
Let us displace infinitesimally the frame $R$ around the origin $\bm A$; let $\overline R$ be its new position; denote by $e^\alpha_i \ (e^\alpha_0 = 0)$ the quantities that define the transition from $R$  to $\overline R$.

The projective infinitesimal transformation considered is a geometric transformation operating on the entire space. Applied to $\bm A$ and $R$ it gives $\bm A'$ and $R'$; applied to $\overline R$ it will give a certain frame $\overline R\,'$ with origin $\bm A'$. 
%
Since the set of two frames $[R', \overline R']$ is deduced by an infinitesimal projective displacement of the set $[R, \overline R]$, we can say that the components of the displacement taking $R'$ onto $\overline R\,'$, are those of the displacement taking $R$ to $\overline R$, that is to say $e\alpha_i$. 
%
The quantities $e^\alpha_i$ are thus not modified by the infinitesimal transformation considered, and if we denote by $d$ the symbol of variation corresponding to this transformation, def = 0 to be the symbol of variation corresponding to this transformation, we can write
\begin{eqnarray*}
d e^\alpha_i = 0.
\end{eqnarray*}
If $\delta$ is the symbol of the variation corresponding to the infinitesimal displacement of the frame $R$ around the origin, the fundamental formulas of structure constructed with the symbols $d$ and $\delta$ (see the first Part, \S 76) now give the infinitesimal variations $\delta \omega^\beta_\alpha$ of the components of the tensor previously introduced.

We have generally
\begin{eqnarray*}
\lefteqn{
\delta \omega^\beta_\alpha (d) - d \omega^\beta_\alpha (\delta)
= \omega^\lambda_\alpha (\delta) \omega^\beta_\lambda (d) 
- \omega^\lambda_\alpha (d) \omega^\beta_\lambda (\delta)
 } \\ && \hspace*{4cm}
 - \delta^\beta_\alpha \left[ \omega^k_0 (\delta) \omega^0_k (d) 
- \omega^k_0 (d) \omega^0_k (\delta) \right] ;
\end{eqnarray*}
here
\begin{eqnarray*}
\omega^\beta_\alpha (d) = \omega^\beta_\alpha, \ \ \ \ 
\omega^\beta_\alpha (\delta) = e^\beta_\alpha, \ \ \ \ 
e^\alpha_0  = 0 ,
 \end{eqnarray*}
from which the formulas we seek
\begin{eqnarray}
\delta \omega^\beta_\alpha = e^\lambda_\alpha \omega^\beta_\lambda - e^\beta_\lambda \omega^\lambda_\alpha + \delta^\beta_\alpha e^0_k \omega^k_0 .
\label{eq:3-5}
\end{eqnarray}































































































































% section complete







































