% !TEX root = [Cartan]-ProjConnection.tex

%  Section completed 17 Sep 2017
\ \\

{\bf 24. Contravariant and covariant analytic vectors. --- }
% 
By specifying appropriately the infinitesimal projective transformation which is at the origin of the present study, we have been able to find the two elementary tensors formed by the contravariant and covariant vectors. The other two elementary tensors, namely the contravariant and covariant analytic vectors, can also be connected to this transformation.

Let us again take up formulas (\ref{eq:3-6}), considering the $\omega^1_0, \omega^2_0, ..., \omega^n_0$ as given quantities (the displacement of the origin of the frame is given). If we only consider only the $\omega^i_i$, we obtain ($i$ being a summation index)
\begin{eqnarray*}
\delta \omega^i_i &=& e^0_i \omega^i_0 + e^k_i \omega^i_k - e^i_k \omega^k_i + n e^0_k \omega^k_0, 
\end{eqnarray*}
which we can write, by a change of indices,
\begin{eqnarray*}
\delta \omega^i_i &=&  (n + 1) e^0_k \omega^k_0 ;
\end{eqnarray*}
we have furthermore
\begin{eqnarray*}
\delta \omega^i_0 &=&  - e^i_k \omega^k_0  . 
\end{eqnarray*}

If we put 
\begin{eqnarray*}
- \frac{ \omega^i_i }{n+1} = X^0, \ \ \ \ \  \omega^k_0 = X^i, 
\end{eqnarray*}
we can write 
\begin{eqnarray*}
\delta X^\alpha &=& - e^\alpha_k X^k ,
\end{eqnarray*}
and we see that the $X^\alpha$ are the components of an analytic contravariant vector, and the class of infinitesimal projective transformations which give to the origin a given displacement and to the quantity
\begin{eqnarray*}
- \frac{ 1}{n+1}(\omega^1_1 +\omega^2_2 + \cdots + \omega^n_n )   
\end{eqnarray*}
a given value.





















































































































































% section complete







































